\chapter*{Samenvatting (in Dutch)}
% TODO END: samenvatting schrijven

Deze thesis gaat over ... \\

Hoofdstuk \ref{ch:Introduction} introduceert enkele belangrijke concepten ... \\

Hoofdstuk \ref{ch:Hotspice} stelt de open-source \hotspice Monte Carlo-simulator voor, die de dynamica van artificieel spin-ijs (ASI) simuleert met als doel om reservoir computing-strategieën te evalueren en optimale systeemparameters te bepalen.
Er wordt een Ising-achtig model gebruikt: elke magneet in het ASI wordt voorgesteld als een punt-dipool die kan wisselen tussen twee stabiele toestanden.
Verschillende varianten op dit model worden voorgesteld, wiens nauwkeurigheid wordt vergeleken met betrekking tot de simulatie van het dynamische gedrag van ASI.
Drie soorten varianten worden besproken:
\begin{itemize}[noitemsep,nolistsep] % With enumitem package
	\item de keuze van Monte Carlo spin-flip-algoritme voor het simuleren van systeemdynamica,
	\item manieren om de eindige grootte van nanomagneten in rekening te brengen voor de berekening van onderlinge magnetostatische interacties,
	\item afschattingen van de effectieve energiebarri\`ere die de twee stabiele toestanden van een nanomagneet scheidt.
\end{itemize}
In de tweede helft van het hoofdstuk worden implementatiedetails besproken, met name de berekening van de magnetostatische interactie tussen alle magneten, evenals andere strategie\"en om de effici\"entie van de Monte Carlo-simulatie te verbeteren.
Het hoofdstuk eindigt met een korte verificatie van de software, door simulaties te vergelijken met analytische oplossingen voor enkele oplosbare systemen. \\

In hoofdstuk \ref{ch:Applications}, ... \\

In hoofdstuk \ref{ch:Conclusion} ...
