\chapter{Applications}\label{ch:Applications}
% TODO END: go through this chapter (and the others, probably) at the end and see if any comments with additional information can be put or merged anywhere.
\glijbaantje{If you find that you're spending almost all your time on theory,\\ start turning some attention to practical things; it will improve your theories.\\ If you find that you're spending almost all your time on practice,\\ start turning some attention to theoretical things; it will improve your practice.}{Donald Knuth}

\begin{adjustwidth}{2em}{2em} % TODO END: update once published.
	\begin{center}
		\textbf{Material from this chapter has also been published in:} \\
	\end{center}
	\vspace{1em}
	\begin{adjustwidth}{0em}{1.5em}
		\begin{itemize}
			\item[\cite{KUR-24}] A.~Kurenkov, J.~Maes, A.~Pac, G.~M.~Macauley, B.~Van~Waeyenberge, A.~Hrabec, and L.~J.~Heyderman.
			\newblock Perpendicular-anisotropy artificial spin ice with spontaneous ordering: a platform for neuromorphic computing with flexible timescales.
			\newblock \emph{ArXiV}, arXiv:\penalty0 2408.12182, 2024.
		\end{itemize}
	\end{adjustwidth}
	\vspace{.5em}
	\begin{center}
		\centering\rule{0.7\linewidth}{0.4pt}
	\end{center}
	\vspace{1em}
\end{adjustwidth}

Our main focus will be on out-of-plane systems. First, however, we will rather briefly discuss several in-plane cases.

\section{In-plane systems}
\subsection{Pinwheel reversal}\label{sec:3:IP_Pinwheel_reversal}
\subsection{Kagome reversal} % Q: move these reversals to the Hotspice chapter?
\subsection{RC by clocking in Pinwheel ASI}

\section{Reservoir computing in out-of-plane systems}
The main factor that attracted attention towards OOP ASI for RC, was that they allow very efficient and relatively simple input and read-out methods to be used.
Early on in the \spinengine project, methods to read the state of OOP ASI had already been demonstrated experimentally on a small scale by ETHZ/PSI.
However, their potential for RC had not yet been investigated.
As \hotspice is well-suited for the simulation of OOP ASI, this presented an appropriate use-case for the software developed in~\cref{ch:Hotspice}. % Furthermore, the other simulator developed by the consortium (flatspin) only supports in-plane magnetisation. Furthermore, \hotspice is more accurate for OOP ASI than IP ASI due to the high degree of symmetry in the former.
\par
Therefore, in this section \hotspice will be used to assess the viability and characteristics of a couple of methods to achieve RC in OOP ASI.
It will be shown that RC is indeed possible in such systems by using an appropriate input and readout protocol.
This section starts with a discussion of the basic properties of OOP ASI and the motivation for researching them.
Then, the section is divided into two main parts, discussing non-volatile and thermally active ASI. % At first, the fabricated OOP ASI were not thermally active, though ASI which spontaneously relaxed to a checkerboard state was achieved later.
\subsection{Characteristics of OOP ASI}
\subsubsection{Motivation} % Input and output
While it is possible to use an external field to switch both IP and OOP nanomagnets, this is undesirable in a real device due to stray fields and the need for bulky magnets.
The main allure of OOP ASI stems from the existence of efficient input and readout mechanisms for such OOP magnets.
Input can be applied via spin-orbit torque (SOT)~\cite{SOT_FM_AFM,SOTswitchingCoPt}, and the system can be read out using the anomalous Hall effect (AHE)~\cite{AHE}.
Both of these only require a current to be passed through an underlayer that is electrically connected to the sample, and were explained earlier in the introduction (\cref{sec:1:ASI_IO}).
% Nonetheless, external fields are often used for input anyway in the lab during the first stages of development, because they are easier to handle than the more complicated methods that would be used in a real device. By the end of the project, Alex had used SOT in his actual devices.

\subsubsection{Structure of OOP nanomagnets} % Co/Pt multilayer, energy contributions
\label{sec:3:OOP_nanomagnet_PMA}
Where IP magnets mostly rely on shape anisotropy to create an easy axis, OOP magnets can be realised using interfacial anisotropy between the ferromagnetic material and the substrate.
One example of this is found at the interface between \ce{Co} and \ce{Pt}, as was used in the experimental OOP ASI fabricated within the context of the \spinengine project.
This \idx{perpendicular magnetic anisotropy} (PMA) acts in the immediate vicinity of the interface, where the magnetisation then prefers to align perpendicular to said interface. \par
Typically, a number of \ce{Co}-\ce{Pt} interfaces are stacked vertically, with the number and thickness of \ce{Co} layers then controlling the OOP anisotropy.
The ferromagnetic layers in such stacks are typically made to be very thin for two main reasons.
Firstly, since the PMA only acts at the interface it does not increase for thicker layers.
Secondly, recall that the energy associated with \xref{shape anisotropy} increases with volume ($\EB=K_\mathrm{u}V$), so a maximal number of layers for a given amount of ferromagnetic material is preferable for PMA to counteract shape anisotropy. \par
Other factors contribute to the anisotropy as well, e.g. a non-linear dependence on the \ce{Pt} layer thickness due to the RKKY interaction~\cite{RKKY_RK,RKKY_K,RKKY_Y}.
However, this coupling is likely not dominant in a \ce{Co}-\ce{Pt} stack, as \ce{Pt} nearly satisfies the Stoner criterion~\cite{PtMagneticOrder}.
Since the \ce{Pt} is combined with a ferromagnetic material (\ce{Co}), it is thought that this induces a ferromagnetic coupling between \ce{Co} layers in spite of the RKKY interaction, for reasonable \ce{Pt} thicknesses~\cite{PerpendicularMagnetizationASI}.

\subsubsection{ASI ground state}
The ground state of any system being used for RC is of utmost importance; it dictates which input methods will induce desirable dynamics in the system and which input protocols will fall flat.
The ground state of most OOP spin ices is some form of anti-ferromagnetic (AFM) ordering, because the dipolar interaction will always encourage opposite magnetisation between any two neighbouring OOP magnets.
Depending on the particular lattice upon which the OOP magnets are placed, this may (e.g. triangle/Cairo; \crefSubFigRef{fig:2:ASIs}{j,l}) or may not (e.g. square/honeycomb; \crefSubFigRef{fig:2:ASIs}{i,k}) induce frustration in the system.
If no frustration is present, the symmetry of the system imposes exactly two opposite ground states.
If frustration is present, no real ground state exists apart from local interactions, though the simple two-fold symmetry will still be present, as in any ASI.
For the OOP square-lattice ASI in particular, which is the lattice we will consider throughout this chapter, the ground state is an AFM checkerboard pattern of `up' ($\uparrow$) and `down' ($\downarrow$) magnetisation.

\subsubsection{Input and readout: experiment to simulation}
\paragraph{Input by spin-orbit torque}
In the experimental system, the intent was to switch the OOP magnets using SOT~\cite{SOTswitchingCoPt}.
Compared to an \xref{external magnetic field}, this allows far greater control over the stimulus applied for each input bit, and can more easily be integrated on-chip without generating stray fields. \par
Since \hotspice does not work with torques, but rather with switching energies, SOT can not directly be modelled as a torque.
However, the net desired effect of SOT in our application is deterministic switching of magnets, which can be achieved in the presence of in-plane symmetry breaking~\cite{SOT_Roadmap}. % SOT_Roadmap mentions symmetry breaking several times, very useful resource
Broken symmetry creates a preferential magnetisation direction, which can be modelled in \hotspice as an additional external field $B_z$.
This is a crude approximation to fit SOT into the Ising-like model; the modelled field $B_z$ is not necessarily related to the field-like SOT torque~\cite{SOT_firstprinciplesCoPt}, though the SOT current density is likely non-linearly related to the modelled field $B_z$.
% In the case without symmetry breaking, the effect may be modelled as a temporary reduction of the PMA, as in reality a sufficiently strong SOT (applied over multiple $\tau_0$) will push the magnetic moment in-plane~\cite{SOT_Roadmap}, after which it falls back randomly (with higher likelihood to the lowest energy state, I suppose). But this approach raises multiple additional questions: is it meaningful to reduce the barrier only slightly?

\paragraph{Readout by anomalous Hall effect}
In our paper~\cite{KUR-24}, \textit{A. Kurenkov} experimentally demonstrated electrical readout of the average magnetisation $\mavg$ using the anomalous Hall effect~\cite{AHE}.
Therefore, $\mavg$ can be used as a magnetic state readout in the simulations.
However, for reservoir computing purposes, multiple readout values are beneficial, so the average magnetisation of each lattice column was used for the readout $y_i(t)$. % Fig. 6a

A discussion about the experimental feasibility of such a grid of local readouts is given in Supplementary Information 7 of our paper~\cite{KUR-24}.
In short, by adding more electrodes to the Hall bar --- the layer underneath the ASI used to measure the magnetic state through the AHE --- inhomogeneities in the current density distribution result in different currents flowing through the different nanomagnets.
This consequently results in unequal contributions to the Hall resistance, providing a way to differentiate between the different magnets.
Individual states of the nanomagnets, not just $\mavg$, can be extracted through a set of linear operations as detailed in the aforementioned Supplementary Information.
In the context of reservoir computing, this means that the use of this readout approach with multiple electrodes can be computationally equivalent to knowing the exact magnetic configuration of the system.


%We will later notice that simple stimuli are not sufficient for RC in this system, and instead more complicated patterns will have to be used to meaningfully address the system. However, the practical implementation of such complicated stimuli requires more complicated current line layouts, which also cannot interfere with the readout. This proves to be a conundrum, which I will address later again.

\subsection{RC in non-volatile OOP square ASI}
\subsubsection{Considerations for devising an appropriate input scheme}
\paragraph{Global field can not work} % Start with schemes that are not appropriate
\paragraph{AFM field can work}
\paragraph{Input encoding} % Binary vs. byte etc. Sigmoid thingy?
\subsubsection{Two-step input scheme: clocking} % With importance of E_B_std

\subsection{Thermally active OOP ASI}
\subsubsection{Energy contributions}
Thermally active OOP ASI are hard to manufacture, because they must strike a delicate balance between several strong energy contributions to end up in an OOP state with a net energy barrier of only several tens of $\kBT$ at most.
To appreciate this difficulty, let us first take a closer look at the various energy contributions in the system --- ranging from local anisotropy to large-scale interactions --- and their dependence on the geometrical structure of the OOP magnets.
\begin{enumerate}
	\item \textbf{Perpendicular magnetic anisotropy} constitutes the largest contribution to the total OOP anisotropy.
	Its interfacial origin and characteristics have been discussed earlier in this chapter (see \cref{sec:3:OOP_nanomagnet_PMA}). % Anisot E: \approx\SI{1.1e-16}{\joule}
	\item \textbf{Shape anisotropy}\indexlabel[nolabel]{shape anisotropy} of each layer due to their \xlabel[nolabel]{demagnetising field}.
	It is of a similar order of magnitude as the PMA, but of opposite sign, favouring an IP magnetisation rather than OOP.
	This contribution was also discussed earlier, see~\cref{sec:2:shape_anisotropy}~and~\ref{sec:3:OOP_nanomagnet_PMA}. % Demag E: \approx\SI{-1.6e-16}{\joule}
	\item \textbf{Inter-layer magnetostatic interaction}.
	The separate FM layers within a single stack also interact, as they can all be considered as separate vertically stacked magnets.
	This vertical stacking provides an additional contribution to the OOP anisotropy because, in general, the magnetisation of separate magnets prefers to point along the axis connecting those magnets.
	One can think of this effect as being similar to the shape anisotropy, but on the scale of separate layers rather than individual atomic magnetic moments.
	This contribution aids the PMA in stabilising the OOP anisotropy, counteracting the \xref{demagnetising field}, but is smaller in magnitude. % Cross MS E: \approx\SI{0.5e-16}{\joule}
	\item \textbf{Inter-magnet magnetostatic interaction} is the only contribution --- besides external input --- that distinguishes between the $\uparrow$ and $\downarrow$ magnetisation states.
	All aforementioned energy contributions are internal to individual OOP magnets, making the \xref{magnetostatic interaction} the sole reason for the AFM ground state of the OOP ASI. % E_MS: \lesssim\SI{1.1e-17}{\joule}
	\item \textbf{Input}.
	In simulations, an \xref{external magnetic field} is used as a proxy to model SOT.
\end{enumerate}
For a system to exhibit spontaneous switching, the first three of these contributions should mostly cancel out --- to the level where their combined influence is of the same order of magnitude as the inter-magnet magnetostatic interaction, which is usually at least an order of magnitude weaker than the other contributions separately.
As we will see in~\cref{sec:3:relaxation}, achieving spontaneous thermal ordering to the ground state may require even finer control.

\paragraph{Magnet geometry}
% The geometry of nanomagnets in an OOP ASI is shown schematically in~\cref{fig:3:OOP_geometry}.
% TODO: make own figure of nanomagnets?

The magnets considered here are round with a diameter $D_\mathrm{NM}$ and consist of $N_{\ce{Co}}$ layers of \ce{Co} with a thickness $t_{\ce{Co}}$, separated and surrounded by \ce{Pt} layers of thickness $t_{\ce{Pt}}$.
They are placed in an ASI where the lateral spacing between magnets is $S_\mathrm{ASI}$ --- resulting in a lattice parameter $a=D_\mathrm{NM}+S_\mathrm{ASI}$ in a square lattice. \par
These five geometrical parameters have a profound effect on the balance between the aforementioned energy contributions, as summarised in~\cref{tab:3:interactions_geometry}.
The polynomial dependence of some of these relations makes the system quite hard to control: a slight manufacturing difference can yield a vastly different net OOP anisotropy, possibly inhibiting the spontaneous formation of a checkerboard ground state.
Note that the magnetostatic interactions depend on the square of the ferromagnetic volume $V=n_{\ce{Co}} t_{\ce{Co}} D_\mathrm{NM}^2$.

\xtable[tab:3:interactions_geometry]{
	Dependence of the energy components present in OOP ASI on the geometrical parameters.
	$x$ represents the parameter above the column.
	The magnet centre-to-centre distance $r = D_\mathrm{NM} + S_\mathrm{ASI}$.
	PMA depends non-trivially on $t_{\ce{Pt}}$.
	Empty cells indicate no significant dependence.
}{
	\begin{tabular}{r|c|c|c|c|c}
		\multicolumn{1}{r}{} & \multicolumn{1}{c}{$t_{\ce{Co}}$} & \multicolumn{1}{c}{$n_{\ce{Co}}$} & \multicolumn{1}{c}{$D_\mathrm{NM}$} & \multicolumn{1}{c}{$t_{\ce{Pt}}$} & \multicolumn{1}{c}{$S_\mathrm{ASI}$} \\
		\hline \hline
		PMA (interfacial anisotropy) &  & $x$ & $x^2$ & $\aquarius$ &  \\
		\hline
		Shape anisotropy (demag) & $x^2$ & $x$ & $x^4$ &  &  \\
		\hline
		Inter-layer MS interaction & $x^2$ & $x^2$ & $x^4$ & $x^{(< -3)}$ &  \\
		\hline
		Inter-magnet MS interaction & $x^2$ & $x^2$ & $x^4$ &  & $r^{(< -3)}$ \\ % S_ASI: $>r^{-3} + \frac{3}{16}D_\mathrm{NM}^2 r^{-5} $
		\hline
		Input (external field) & $x$ & $x$ & $x^2$ &  &  \\
		\hline
	\end{tabular}
}

These five geometrical parameters are bounded by several physical and practical constraints.
Note that the numerical values of the limits given below are only indicative and may depend on the exact values of the other geometrical parameters.
\begin{itemize}
	\item \textbf{FM layer thickness} $\boldsymbol{t_{\ce{Co}}} \lesssim \SI{1.45}{\nano\metre}$. \newline
	This upper bound was experimentally determined~\cite{KUR-24} and is imposed by shape anisotropy.
	Layers must be sufficiently thin to maintain OOP anisotropy, as \xref{shape anisotropy} is proportional to volume (and hence $\propto t_{\ce{Co}}$) while the PMA is mostly independent of $t_{\ce{Co}}$. \par % t_{\ce{Co}} controls the demag compensation of $K_u$.
	The closer the thickness is to this upper bound, the lower the effective OOP anisotropy will be.
	Therefore, non-volatile ASI must steer clear of this limit, while thermally active ASI generally try to approach this limit as close as possible.
	As such, this geometrical parameter is often key for balancing the energy landscape.
	\item \textbf{Separation between magnets} $\boldsymbol{S_\mathrm{ASI}} \gtrsim \SI{20}{\nano\metre}$. \newline
	This constraint is of a practical nature, as the minimal size of lateral geometrical features is limited by the accuracy of the lithographic process. \par
	The separation only affects the strength of the magnetostatic interaction between neighbouring magnets, as approximated in \hotspice by \cref{eq:2:E_MS} and the second-order correction of \cref{eq:2:E_MS_order2}.
	In ASI, a strong MS coupling --- hence low $S_\mathrm{ASI}$ --- is often preferred.
	However, for thermally active ASI we will soon see that excessively strong MS coupling is detrimental, resulting in a ``sweet spot'' of MS coupling energies.
	\item \textbf{Number of layers} $\boldsymbol{n_{\ce{Co}}} \lesssim 8$. \newline
	This limit is of a practical nature as well --- property drift during deposition of successive layers makes it harder to achieve small $S_\mathrm{ASI}$ for increasingly tall stacks. \par
	Note that the number of layers does not affect the balance between PMA and shape anisotropy, as the number of interfaces per amount of magnetic material remains constant.
	However, since MS interactions grow $\propto n_{\ce{Co}}^2$, the inter-layer coupling --- and with it the net OOP anisotropy --- becomes increasingly significant for a large number of layers.
	\item \textbf{Diameter} $\boldsymbol{D_\mathrm{NM}} \lesssim \SI{200}{\nano\metre}$. \newline
	Magnets wider than this limit were observed to take on multi-domain states~\cite{KUR-24}, as their demagnetisation energy becomes dominant due to its rapid $\propto D_\mathrm{NM}^4$ growth.
	Even though in-plane magnets without uniform magnetisation have been used for computation~\cite{gartside2022reconfigurable}, we did not intend to use this in the OOP systems.
	Furthermore, the \hotspice simulator requires a single-domain magnetisation state for its Ising-like model to be applicable. \par
	The diameter can be used to control the relative strength between PMA and shape anisotropy and greatly affects the inter-magnet MS coupling.
	\item \textbf{Spacer layer thickness} $\boldsymbol{t_{\ce{Pt}}} \gtrsim \SI{0.4}{\nano\metre}$. \newline
	Thinner layers may become discontinuous, which is detrimental to the PMA.
	On the other hand, a thin spacer layer maximises the net OOP anisotropy due to the $\approx t_{\ce{Pt}}^{-3}$ inter-layer MS coupling dependence.
	Note that $t_{\ce{Pt}}$ must still be chosen appropriately to promote FM coupling between layers while maintaining the interfacial anisotropy. % Accounting for various effects such as RKKY coupling
\end{itemize}

In the particular case of the thermally active OOP ASI fabricated within the \spinengine project, the nanomagnets consist of $n_{\ce{Co}}=7$ ferromagnetic layers of $D_\mathrm{NM}=\SI{170}{\nano\metre}$ diameter and $t_{\ce{Co}} \approx \SI{1.45}{\nano\metre}$ thickness, with $t_{\ce{Pt}} \approx \SI{0.8}{\nano\metre}$ spacing layers in between.
The edge-to-edge spacing between neighbouring magnets is \SI{30}{\nano\metre}, unless specified otherwise, with a saturation magnetisation $M_\mathrm{sat}=\SI{1063}{\kilo\ampere\per\metre}$~\cite{Msat_Co}.

\subsubsection{Relaxation characteristics}
To better inform our choices for RC in thermally active ASI, it is essential to familiarize ourselves with its spontaneous relaxation process.
In this section, we numerically investigate this process with \hotspice by initializing an OOP ASI in a uniform ($\uparrow$) state and subsequently observing its decay towards the checkerboard ground state over a certain timespan.
In the Ising-like model used by \hotspice, two essential system parameters remain, denoted as
\begin{itemize}[leftmargin=4.1em]
	\item[$\boldsymbol{\EEA}$ ---] the energy barrier of a single non-interacting OOP magnet due to the net OOP anisotropy.
	This is the combined contribution of PMA, shape anisotropy and the inter-layer magnetostatic interaction.
	\item[$\boldsymbol{\EMC}$ ---] the magnetostatic coupling energy between nearest neighbours.
	This is responsible for the relaxation to the ground state.
\end{itemize}
Due to the N\'eel-Arrhenius switching law~\eqref{eq:2:Néel}, it is more appropriate to write dimensionless energies ($\EEA/\kBT$ and $\EMC/\kBT$), as it is only this ratio --- not the absolute value of the energies or temperature --- that controls the switching times. \par
In real ASI, manufacturing inevitably introduces imperfections.
Slight geometrical variations can result in a significant variance of the net OOP anisotropy between magnets.
In simulations, this will be modelled by sampling $\EEA$ from a normal distribution: unless otherwise specified, a standard deviation $\sigma(\EEA) = \SI{5}{\percent}$ is used.
This provides pinning sites for domain walls, making the relaxation process more reproducible: such ``consistent variation'' within an ASI could ultimately be beneficial for RC by providing a richer output space. \par
The effective energy barrier $\EBeff$ of a magnet $i$ can then symbolically be written as
\begin{equation}
	\label{eq:3:OOP_relaxation_EBeff}
	\EBeff{}_{,i}(t) = \EEA \Big(1 + \sigma\ab(\EEA) \hat{\chi}_i \Big) + \EMC \sum_j s_i(t) D_{ij} s_j(t)
\end{equation}
with $\hat{\chi}_i$ a random value from a standard normal distribution and $D_{ij}$ a factor proportional to the magnetostatic interaction energy between magnets $i$ and $j$. \\\par

The relaxation process can be tracked by the average magnetisation $\mavg$ and the local antiferromagnetic parameter $\qNN = (1 - \langle s_i s_{i+1} \rangle)/2$.
In the uniform state --- where all magnetic moments point `up' --- their values are $\mavg=1$ and $\qNN=0$.
During relaxation, they transition to $\mavg=0$ and $\qNN=1$, corresponding to a checkerboard ground state.
Note that, while for a random state $\qNN \approx 0.5$, a value of $\qNN=0.5$ does not necessarily imply a random state. \par

\cref{fig:3:OOP_relaxation} shows relaxation profiles for a few combinations of $\EEA$ and $\EMC$, providing some insight into the relaxation process.
Each panel shows the mean, standard deviation and 1st/99th percentile of $\mavg$ and $\qNN$ over 200 relaxations --- this variance is both due to the randomness of N\'eel-Arrhenius switching and because each relaxation used unique randomly sampled $\EEA$ for individual magnets, in accordance with the $\sigma(\EEA) = \SI{5}{\percent}$ variation.
Each relaxation consisted of at most $40N$ switches using the first-switch method, as this is sufficient for the system to achieve equilibrium.
The size of the ASI does not significantly affect the relaxation process, apart from reducing the spread on these traces.

\xfig{3_OOP/OOP_relaxation.pdf}{
	\label{fig:3:OOP_relaxation}
	Dependence of the ordering timescale on the energy balance between net OOP anisotropy $\EEA$ and nearest-neighbour magnetostatic coupling $\EMC$ in an $11 \times 11$ OOP square-lattice ASI. % This system size was chosen because it matched the size of our fabricated systems discussed later.
	The ASI is initialised in a uniform magnetic state with all spins pointing `up' and is released at $t = 0$, after which the lattice relaxes to lower energies.
	Blue traces show the absolute value of the average magnetisation $\mavg$ while orange traces represent the local antiferromagnetic parameter $\qNN$.
	Each trace shows the mean (central line), standard deviation (central shaded area), and 1st/99th percentile (outer shaded area) of 200 repeated relaxations with randomly sampled $\EEA$ profiles.
	At most $40N$ switches are simulated for each relaxation.
	The green shaded region highlights the second stage of ordering.
	The simulations were stopped after \SI{e4}{\second} and the resulting final magnetic states are shown in \textbf{the insets}, where white (black) indicates up (down) spins.
}

\paragraph{Logarithmic relaxation}
Most strikingly, by using a logarithmic scale for the time axis, both $\mavg$ and (to a lesser extent) $\qNN$ trace a very straight line during relaxation.
Because of this \idx{logarithmic relaxation}, the first switches occur on a vastly different timescale than the last --- each subsequent switch occurs on average $x$ times later than the previous switch ($t_{i+1} = x t_i \Rightarrow t_n = t_0 x^n$). \par
The cause of this is the magnetostatic interaction\footnote{
	Without the \xref{magnetostatic interaction}, the system would simply be a non-interacting ensemble of nanomagnets, which is known to relax to the ground state exponentially.
	This is the opposite of logarithmic relaxation; quantities like $\mavg$ would only show up as straight lines when using a linear time axis and logarithmic y-axis.
	% Exponential decay would be much easier to work with for temporal input, but having no MS coupling defeats the point of using an ASI for some nonlinearity.
}.
Initially, all magnets are in a high-energy state as they all point in the same direction.
After a certain amount of time, a first magnet will switch, thereby slightly stabilising all other magnets in the ASI to some degree according to the second term of~\cref{eq:3:OOP_relaxation_EBeff}.
Whichever magnet randomly switches next will therefore experience a slightly higher effective energy barrier $\EBeff$ than before.
On average, this means that the effective barrier $\max(\EBeff)$ of the most unstable magnets in the system --- those most eager to switch next --- will increase linearly with the amount of switched magnets.
The N\'eel-Arrhenius equation~\eqref{eq:2:Néel} translates this to an exponentially increasing switching time. \\\par

\cref{fig:3:OOP_relaxation} also shows that the slope generally becomes less steep for higher coupling $\EMC$.
Following the previous discussion, we now understand that high coupling results in a greater decrease of $\EBeff$ per switch, and consequently a larger exponential increase in the switching time.
Meanwhile, the net OOP anisotropy $\EEA$ only affects the timescale of the relaxation --- it does not affect the slope, but simply shifts the relaxation curves along the (logarithmic) time axis.
Therefore, to achieve an ASI that exhibits a particular relaxation timescale, accurate control of $\EEA$ on a scale of $\sim 10\kBT$ is required. \par
The effects of $\EMC$ and $\EEA$ can be quantified by fitting a logarithmic curve to the mean relaxation curves of the nine panels in ~\cref{fig:3:OOP_relaxation}, yielding the general trends
\begin{equation}
	\mavg = a \log_{10}(t \nu_0) + b \quad \mathrm{with} \quad \begin{cases}
		a \approx -\frac{\kBT}{2\EMC} \, \mathrm{,} \\
		b \approx \frac{\kBT}{6\EMC}(\frac{\EEA}{\kBT} - 1.5) - 0.05 \, \mathrm{.}
	\end{cases}
\end{equation} % TODO: more rigorous fit?

% Note that the change in slope $a$ and the offset $b$ compensate to some extent such that the final switches ($\mavg \rightarrow 0$) occur around the same time nearly independently of $\EMC$.
% The intercept of the fit for $\mavg=0$ is indeed far less sensitive to $\EMC$ than to $\EEA$:
% \begin{equation}
%  	t_{\mavg=0} = \frac{10^{-b/a}}{\nu_0} = \frac{10^{\frac{\EEA - 0.3 \EMC}{3\kBT} - 0.5}}{\nu_0} \mathrm{.}
% \end{equation}

\paragraph{Two-stage relaxation}
While $\mavg$ follows a near-perfect logarithmic relaxation, $\mavg \approx 0$ is reached earlier than the perfect checkerboard ordering $\qNN = 1$ --- most strikingly in panels 6 and 9 of~\cref{fig:3:OOP_relaxation}.
This is accompanied by a decreased ordering rate $\frac{\partial \qNN}{\partial \log(t)}$, marking the transition between stage $\circled{1}$ and $\circled{2}$ of relaxation.
The latter is indicated by the green shaded area in the figure. \par
This can be understood by considering the microstate of the ASI: to this end, the final state of each panel in~\cref{fig:3:OOP_relaxation} is shown as an inset.
Panels 1 and 2 show the system during the logarithmic relaxation of both $\mavg$ and $\qNN$ in stage $\circled{1}$.
Here, many magnets still have three or four NN with the same magnetisation state, causing random magnets to switch and small checkerboard domains to nucleate throughout the system.
Because these domains are twofold degenerate, the perfect checkerboard ordering is not reached directly.
Instead, expanding domains are likely to be separated by domain walls, as seen in panels 3 and 4 which have just barely reached stage $\circled{2}$.
From this point on, a further increase in $\qNN$ can only\footnote{
	If the magnetostatic coupling $\EMC$ is insufficient, a perfect checkerboard ordering will not be achieved and $\qNN$ may not increase much further --- in panel 7 of~\cref{fig:3:OOP_relaxation}, $\qNN$ settles around $\approx 0.8$.
} be achieved by switching magnets that are part of a domain wall.
However, even though domain wall magnets are the least stable magnets in the system, they still typically have 3 stabilising NN, resulting in a significant effective energy barrier that depends on both $\EEA$ and $\EMC$.
Furthermore, when a domain wall magnet finally switches, it simply moves the domain wall by one lattice unit without significantly altering the macrostate of the system.
This is in stark contrast with stage $\circled{1}$, where the macrostate was constantly changing.
Only when the domain wall randomly encounters a domain wall of opposite polarity or an edge of the system, can the total length of domain walls in the system decrease --- as occurs between panels 5 and 8.
This process is more akin to a random walk, and therefore has a polynomial time dependency rather than a logarithmic one.
For this reason, stage $\circled{2}$ shows up in~\cref{fig:3:OOP_relaxation} as a sudden increase of $\qNN$ after a plateau, as most clearly seen in panel 9. \par
Once the system has finally approached the checkerboard state with $\mavg \rightarrow 0$ and $\qNN \rightarrow 1$, random thermal fluctuations will still cause small excursions to $\mavg > 0$ and $\qNN < 1$.
These are generally more pronounced for low coupling $\EMC$, as can clearly be seen in the $\EEA = 20\kBT$ row of~\cref{fig:3:OOP_relaxation}.
This is because $\EMC$ is the only factor responsible for order in the system: low coupling makes magnets less inclined to reside in the ground state, giving the opportunity for multiple magnets to be in an unstable state at once and drive $\mavg$ further from 0.

%\paragraph{Miscellaneous remarks}
%During the relaxation process, one may be inclined to draw the conclusion from~\cref{fig:3:OOP_relaxation} that larger $\EMC$ yields more random variation.
%This is mostly an illusion due to the steep slope of the relaxation for low $\EMC$: while the lower slope for $\mavg$ increases the random variation along the temporal axis, the vertical spread actually decreases for larger $\EMC$. \par

\paragraph{Phase diagram}
A stronger coupling $\EMC$ causes faster decay of $\mavg \rightarrow 0$: in panels 4--6 of~\cref{fig:3:OOP_relaxation}, the time to reach $\mavg \sim 0$ decreases from $\approx \SIrange{e3}{10}{\second}$.
However, strong coupling does not necessarily result in faster ordering $\qNN \rightarrow 1$ --- as panels 8 and 9 clearly illustrate --- because domain wall magnets typically have 3 oppositely magnetised nearest neighbours, resulting in a net stabilising effect proportional to $\EMC$ (and $\EEA$).
Therefore, for a system to reach the checkerboard state on a given timescale, a sufficiently low $\EMC$ and $\EEA$ are required. \par
This is most clearly seen in phase diagrams like~\cref{fig:3:OOP_relaxation_continuous}, which summarise the values of $\mavg$ and $\qNN$ at $t = \SI{e4}{\second}$ (i.e., the maximum time in \cref{fig:3:OOP_relaxation}) over a wide range of $\EEA$ and $\EMC$.
This encompasses the nine $(\EMC,\EEA)$ combinations of the various panels in~\cref{fig:3:OOP_relaxation}, as indicated by grey numbered crosses in~\cref{fig:3:OOP_relaxation_continuous}. \par
Five distinct regions can be distinguished.
In region $\mathrm{I}$ (panel 1), weak coupling and high anisotropy result in a non-volatile system --- as used in the previous section --- where no spontaneous switches occur and $\mavg = 1$ and $\qNN = 0$ remain unchanged.
Increasing the $\EMC/\EEA$ ratio shifts the system though the transient region $\mathrm{II}$ (panel 2), characterised by decreasing $\mavg$ and increased ordering.
Further increase of $\EMC/\EEA$ shifts the system to region $\mathrm{III}$ (panel 6) with $\mavg \approx 0$ but $\qNN \neq 1$, characteristic of stage $\circled{2}$.
Any further increase of $\EMC/\EEA$ does not affect the average magnetisation or ordering at $t = \SI{e4}{\second}$ --- unless both $\EEA$ and $\EMC$ are sufficiently small, as in regions $\mathrm{IV}$ and $\mathrm{V}$.

\xfig{3_OOP/OOP_relaxation_continuous.pdf}{
	\label{fig:3:OOP_relaxation_continuous}
	Phase diagrams of the average magnetisation $\mavg$ and the local antiferromagnetic parameter $\qNN$ as a function of net out-of-plane anisotropy $\EEA$ and nearest-neighbour magnetostatic coupling $\EMC$ at $t=\SI{e4}{\second}$.
	Five regions can be distinguished: $\mathrm{I}$ -- frozen state, $\mathrm{II}$ -- transient states, $\mathrm{III}$ -- state with domains and domain boundaries, $\mathrm{IV}$ -- checkerboard ordering, $\mathrm{V}$ -- superparamagnetic state.
	The white and black dotted line respectively correspond to $\mavg = 0.05$ and $\qNN = 0.95$.
	The nine panels of~\cref{fig:3:OOP_relaxation} are indicated by grey crosses.
	The three symbols correspond to the experimental lattices shown in~\cref{fig:3:OOP_relaxation_fit} with nanomagnet diameter $D_\mathrm{NM} = \SI{170}{\nano\metre}$, \ce{Co} thickness $t_{\ce{Co}} = \SI{1.45}{\nano\metre}$ and nanomagnet separations $S_\mathrm{ASI} = \SI{20}{\nano\metre}$ (circle), $S_\mathrm{ASI} = \SI{25}{\nano\metre}$ (triangle), $S_\mathrm{ASI} = \SI{30}{\nano\metre}$ (cross). % TODO: create and reference that figure
}

Region $\mathrm{IV}$ (panel 9) is the only region where the system spontaneously achieves the checkerboard ground state.
In~\cref{fig:3:OOP_relaxation_continuous}, it is delineated by the dotted black line that envelops a region with $\qNN \geq 0.95$.
Reducing $\EMC$ too much puts the system in region $\mathrm{V}$ (panel 7), where the system is superparamagnetic at this timescale.
For $t = \SI{e4}{\second}$, region $\mathrm{V}$ extends upwards to $\EEA \lesssim 32 \kBT$, since the N\'eel-Arrhenius switching law states that nanomagnets with such anisotropy have a mean switching time $\lesssim \exp(32)/\nu_0 \approx \SI{e4}{\second}$.
The main region of interest, region $\mathrm{IV}$ with $\qNN \approx 1$, is therefore constrained from the left by the superparamagnetic regime, from the top by the transient/frozen regime, and from the right by states in which domain boundaries are too stable at the given timescale. \par
The phase diagrams in~\cref{fig:3:OOP_relaxation_continuous} are snapshots at $t = \SI{e4}{\second}$ and, with increasing time, region $\mathrm{IV}$ will expand and the boundaries between the different regions will shift upwards.
Therefore, if the observation is long enough and $\EMC$ is non-negligible, a system located in the frozen region at $t = \SI{e4}{\second}$ may eventually find itself in the more ordered Regions $\mathrm{II}$, $\mathrm{III}$ or $\mathrm{IV}$.
However, due the exponential character of the N\'eel-Arrhenius law, this shift will be minimal and region $\mathrm{IV}$ will not become much bigger than in~\cref{fig:3:OOP_relaxation_continuous} for any realistic timescale~\cite{KUR-24}.


\paragraph{Summary}
The state of an OOP ASI is defined by the net OOP anisotropy $\EEA$, magnetostatic coupling $\EMC$ and elapsed time $t$.
Increasing $\EMC$ causes the relaxation to occur over a larger amount of timescales.
Meanwhile, $\EEA$ controls the timescale of the relaxation but does not affect the amount of timescales over which the relaxation occurs.
For reservoir computing purposes, a sufficiently low $\EEA$ and $\EMC$ are required --- but not too low, to maintain OOP anisotropy and a ground-state checkerboard ordering, respectively.
Therefore, practical applications exploiting systems with specific dynamics will require careful adjustment of the lattice and nanomagnet dimensions~\cite{KUR-24}.

\paragraph{Exchange-coupled system} % TODO: expand, could be omitted if not sufficiently worked out

\subsubsection{Fitting MFM images}\label{sec:3:OOP:MFM}
% TODO: extend this intro by saying that simple calculations based on the cosines and sines of the four interactions (recall: this was calculated by Alex using Mathematica) do not give a thermally active energy barrier. See the following comment for more details.
% In absence of the neighbors (i.e. no \code{externalddE}, their combined energy barrier is $\approx\SI{-3.6e-18}{\joule}=\SI{-22.5}{\electronvolt}=-871k_BT$ which is far too large to get thermal switching, and on top of that it is negative which means that the magnet would rather sit in-plane which is obviously not observed in reality. Curiously, Alex observes thermal relaxation nonetheless. The energy barrier discrepancy between Mathematica and reality could be due to non-coherent reversal, resulting in a lower energy barrier in reality.
% However, Alex said that the energy barriers calculated with Mathematica in this manner did not seem to be very accurate: in the simulation, they were several tens of \SI{}{\electronvolt} (read: many $k_B T$), while in reality he saw thermal relaxation upon removal of an external saturating stimulus.
\paragraph{Bayesian optimisation}
\paragraph{Results} % E_B etc.

\subsection{RC in thermally active OOP square ASI}
\subsubsection{Concept} % Lessons learnt from non-volatile ASI
\paragraph{Input torques: representation in \hotspice}
Contrary to the binary input in the non-volatile OOP ASI, thermally active ASI can be addressed by a continuous range of input.
The input --- in reality applied through spin-orbit torque --- was applied in the simulations in the form of an OOP magnetic field $B_z(t)$ acting on the entire system.
The field was scaled such that its OOP component extends from $B_0$ (minimum) to $B_1$ (maximum), i.e. $B_0 \leq B_z \leq B_1$.

\paragraph{Readout method}
The inverse mean squared error (1/MSE) between o(t) and the desired result (the sawtooth) was used as a performance metric.

\subsubsection{RC: signal transformation}
\paragraph{Motivation} % Why do signal transformation? (scalar, nonlinear, memory...)
\paragraph{Baseline}
\paragraph{Parameter dependencies}
\subparagraph{Field magnitude}
\subparagraph{Frequency}
\subparagraph{Gradient}
\subparagraph{System size and readout resolution}
\subparagraph{Exchange/no exchange}
\subparagraph{Transformation}
\paragraph{Bayesian optimisation: Mackey-Glass}
\paragraph{Reflecting on results} % Is the performance decent? Compare to literature. MSE is not an ideal parameter, should use something that accounts more for the general shape.
