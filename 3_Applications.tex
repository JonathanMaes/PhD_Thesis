\chapter{Applications}\label{ch:Applications}
%\glijbaantje{Quote}{Person}

Our main focus will be on out-of-plane systems. First, however, we will rather briefly discuss several in-plane cases.

\section{In-plane systems}
\subsection{Pinwheel reversal}
\subsection{Kagome reversal}
\subsection{RC by clocking in Pinwheel ASI}

\section{Out-of-plane systems}
\subsection{Characteristics of OOP ASI}
\subsubsection{Motivation} % Input and output
\subsubsection{Fabrication} % Co/Pt multilayer, energy contributions
\subsubsection{Characteristics} % Ground state is AFM etc.

\subsection{RC in frozen OOP square ASI}
\subsubsection{Considerations for devising an appropriate input scheme}
\paragraph{Global field can not work} % Start with schemes that are not appropriate
\paragraph{AFM field can work}
\paragraph{Input encoding} % Binary vs. byte etc. Sigmoid thingy?
\subsubsection{Two-step input scheme: clocking} % With importance of E_B_std

\subsection{RC in thermally active OOP square ASI}
\subsubsection{Concept} % Lessons learnt from frozen ASI
\paragraph{Input and output torques: representation in Hotspice}
\paragraph{Readout method}
\subsubsection{Logarithmic relaxation}
\paragraph{Relaxation}
\paragraph{Fitted values for logarithmic behavior}
\paragraph{Exchange-coupled system} % Could be omitted if not sufficiently worked out
\subsubsection{Fitting MFM images}\label{sec:3:OOP:MFM}
\paragraph{Bayesian optimization}
\paragraph{Results} % E_B etc.
\subsubsection{RC: signal transformation}
\paragraph{Motivation} % Why do signal transformation? (scalar, nonlinear, memory...)
\paragraph{Baseline}
\paragraph{Parameter dependencies}
\subparagraph{Field magnitude}
\subparagraph{Frequency}
\subparagraph{Gradient}
\subparagraph{System size and readout resolution}
\subparagraph{Exchange/no exchange}
\subparagraph{Transformation}
\paragraph{Bayesian optimization: Mackey-Glass}
\paragraph{Reflecting on results} % Is the performance decent? Compare to literature. MSE is not an ideal parameter, should use something that accounts more for the general shape.
