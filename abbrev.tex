\chapter{List of abbreviations}
{ % TODO END: go through this list and explain any unused abbreviations in the main text if applicable, otherwise remove. Then go through main text and check if all abbreviations are here.
	% TODO END: check if all these abbreviations are written in full correctly throughout the text, e.g. search for anti-ferromagnetic and change it to antiferromagnetic (or however I decide to write it). Be consistent with dashes.
	 	% TODO END: label the final list of abbreviations so all abbreviations appear in the index with easy links here to their first occurrence in the text.
    \addtolength{\skip\footins}{1pc}
    \begin{longtable}[l]{ll}
        AFM   & Antiferromagnetic                          \\
        AHE   & Anomalous Hall effect                       \\
        AMR   & Anisotropic magnetoresistance               \\
        ANN   & Artificial neural network                   \\
        %AP    & Anti-parallel (when referring to neighbouring spins) \\
        ASI   & Artificial spin ice                         \\
        BC    & Boundary conditions                         \\
        BO    & Bayesian optimisation                       \\
        %BPDC  & Backpropagation decorrelation               \\
        BPTT  & Backpropagation through time                \\
        %CMOS  & Complementary metal-oxide-semiconductor     \\
        CPU   & Central processing unit                     \\
        CSD   & Critical slowing down                       \\
        %DW    & Domain wall                                 \\
        EA    & Effective anisotropy                        \\
        ESN   & Echo state network                          \\
        % ETHZ  & Eidgen\"ossische Technische Hochschule Z\"urich \\
        FD    & Finite-difference                           \\
        %FFT   & Fast Fourier transform                      \\
        %FM    & Ferromagnetic                               \\
        FMR   & Ferromagnetic resonance                     \\
        FNN   & Feedforward neural network                  \\
        FPGA  & Field-programmable gate array               \\
        %GMR   & Giant magnetoresistance                     \\
        GPU   & Graphics processing unit                    \\
        IP    & In-plane                                    \\
        KMC   & Kinetic Monte Carlo                         \\
        %LIF   & Leaky integrate-and-fire                    \\
        % LLG   & Landau-Lifshitz-Gilbert                     \\
        LMC   & Linear memory capacity                      \\
        LSM   & Liquid state machine                        \\
        MC    & Magnetostatic coupling\footnote{Not to be confused with ``Monte Carlo'', which is written in full unless it is part of a longer abbreviation like KMC, MCMC or MCS.} \\
        MCMC  & Markov chain Monte Carlo                    \\
        MCS   & Monte Carlo sweep (i.e., 1 Monte Carlo step per magnet) \\
        MEMS  & Micro-electronic mechanical systems         \\
        MFM   & Magnetic force microscopy                   \\
        MG    & Mackey-Glass                                \\
        MOKE  & Magneto-optical Kerr effect                 \\
        MRAM  & Magnetic random-access memory               \\
        MS    & Magnetostatic                               \\
        MSE   & Mean squared error                          \\
        MTJ   & Magnetic tunnel junction                    \\
        MTXM  & Magnetic transmission X-ray microscopy      \\
        NL    & Non-linearity                               \\
        NN    & Nearest neighbour\footnote{Second-nearest neighbours are abbreviated as 2NN; further neighbours as 3NN, etc.} \\
        % NTNU  & Norges teknisk-naturvitenskapelige universitet \\
        OLS   & Ordinary least-squares                      \\
        OOP   & Out-of-plane                                \\
        PBC   & Periodic boundary conditions                \\
        PC    & Parity check                                \\
        PEEM  & Photo-emission electron microscopy          \\
        PMA   & Perpendicular magnetic anisotropy           \\
        PRC   & Physical reservoir computing                \\
        % PSI   & Paul Scherrer Institute                      \\
        RAM   & Random-access memory                        \\
        RC    & Reservoir computing                         \\
        RKKY  & Ruderman-Kittel-Kasuya-Yosida interaction   \\
        RNG   & Random number generator                     \\
        RNN   & Recurrent neural network                    \\
        %RNR   & Rotating neurons reservoir                  \\
        %RTRL  & Real-time recurrent learning                \\
        SJG   & Stratified jittered grid                    \\
        SOT   & Spin--orbit torque                           \\
        STO   & Spin-torque oscillator                      \\
        STT   & Spin-transfer torque                        \\
        %TA    & Task-agnostic                               \\
        %VLSI  & Very large-scale integration                \\
        XMCD  & X-ray magnetic circular dichroism           \\
        XOR   & Exclusive or                                \\
        YIG   & Yttrium iron garnet                         \\
    \end{longtable}
}
