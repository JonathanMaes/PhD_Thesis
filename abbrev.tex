\chapter{List of abbreviations}
{ % TODO END: go through this list and explain any unused abbreviations in the main text if applicable, otherwise remove. Then go through main text and check if all abbreviations are here.
    \addtolength{\skip\footins}{1pc}
    \begin{longtable}[l]{ll}
        AFM   & Anti-ferromagnetic                          \\
        AHE   & Anomalous Hall effect                       \\
        ANN   & Artificial neural network                   \\
        AP    & Anti-parallel (when referring to neighbouring spins) \\
        ASI   & Artificial spin ice                         \\
        BC    & Boundary conditions                         \\
        BO    & Bayesian optimisation                       \\
        BPDC  & Backpropagation decorrelation               \\
        BPTT  & Backpropagation through time                \\
        CMOS  & Complementary metal-oxide-semiconductor     \\
        CPU   & Central processing unit                     \\
        CR    & Cyclic reservoir                            \\
        CSD   & Critical slowing down                       \\
        DSP   & Digital signal processing                   \\
        DW    & Domain wall                                 \\
        ESN   & Echo state network                          \\
        % ETHZ  & Eidgen\"ossische Technische Hochschule Z\"urich \\
        FD    & Finite difference                           \\
        FFT   & Fast Fourier transform                      \\
        FM    & Ferromagnetic                               \\
        FMR   & Ferromagnetic resonance                     \\
        FNN   & Feedforward neural network                  \\
        FPGA  & Field-programmable gate array               \\
        GMR   & Giant magnetoresistance                     \\
        GPU   & Graphics processing unit                    \\
        IP    & In-plane                                    \\
        KMC   & Kinetic Monte Carlo                         \\
        LEEM  & Low-energy electron microscopy              \\
        LIF   & Leaky integrate-and-fire                    \\
        % LLG   & Landau-Lifshitz-Gilbert                     \\
        LRO   & Long-range order                            \\
        LSM   & Liquid state machine                        \\
        MC    & Memory capacity\footnote{Not to be confused with ``Monte Carlo'', for which the abbreviation KMC is used.} \\
        MCMC  & Markov chain Monte Carlo                    \\
        MCS   & Monte Carlo sweep (1 Monte Carlo step per magnet) \\
        MEMS  & Micro-electronic mechanical systems         \\
        MEP   & Minimum energy path                         \\
        MFCC  & Mel-frequency cepstral coefficients         \\
        MFM   & Magnetic force microscopy                   \\
        MG    & Mackey-Glass                                \\
        ML    & Machine learning                            \\
        MLP   & Multi-layer perceptron                      \\
        MOKE  & Magneto-optical Kerr effect                 \\
        MSE   & Mean squared error                          \\
        NARMA & Non-linear auto-regressive moving average   \\
        % NEB   & Nudged elastic band                         \\
        NL    & Non-linearity                               \\
        NN    & Nearest neighbour                           \\
        NNN   & Next nearest neighbour\footnote{Also abbreviated as 2NN; further neighbours as 3NN, etc.} \\
        NRMSE & Normalised root-mean-square error           \\
        % NTNU  & Norges teknisk-naturvitenskapelige universitet \\
        OLS   & Ordinary least-squares                      \\
        OoM   & Order of magnitude                          \\
        OOP   & Out-of-plane                                \\
        P     & Parallel (when referring to neighbouring spins) \\
        PBC   & Periodic boundary conditions                \\
        PEB   & Plain elastic band                          \\
        PEEM  & Photo-emission electron microscopy          \\
        PMA   & Perpendicular magnetic anisotropy           \\
        PSI   & Paul Scherrer Institut                      \\
        RC    & Reservoir computing                         \\
        RNN   & Recurrent neural network                    \\
        RNR   & Rotating neurons reservoir                  \\
        RTRL  & Real-time recurrent learning                \\
        SEM   & Scanning electron micrograph                \\
        SHE   & Spin Hall effect                            \\
        SI-1  & Spin ice 1 (a kagome ASI phase)             \\
        SI-2  & Spin ice 2 (a kagome ASI phase)             \\
        SJG   & Stratified jittered grid                    \\
        SOT   & Spin-orbit torque                           \\
        STO   & Spin-torque oscillator                      \\
        STT   & Spin-transfer torque                        \\
        % SW    & Stoner-Wohlfarth                            \\
        TA    & Task-agnostic                               \\
        % UHV   & Ultrahigh vacuum                            \\
        VLSI  & Very large-scale integration                \\
        XMCD  & X-ray magnetic circular dichroism           \\
    \end{longtable}
}
