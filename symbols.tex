\chapter{List of constants and symbols}  % TODO END: put all symbols in this list
\section*{List of constants}
\begin{longtable}[l]{p{60pt} p{140pt} p{200pt}}
	$e$ & Electron charge & \SI{1.6022e-19}{\coulomb} \\
	% $\gamma$&gyromagnetic ratio&$1.76086\times 10^{11}$\,rad/Ts\\
	% $\gamma_0$&$\mu_0\gamma$&$2.21\times\,10^5$\,m/As\\
	$k_\mathrm{B}$ & Boltzmann constant & \SI{1.3087e-23}{\joule\per\kelvin} \\
	$\mu_0$ & Vacuum permeability & $4 \pi \times \SI{e-7}{\tesla\metre\per\ampere}$ \\
	% $\mu_\mathrm{B}$&Bohr magneton&$9.274\times 10^{-24}$\,Am$^2$\\
\end{longtable}

\section*{List of symbols}
\subsection*{Mathematical operators}
\begin{longtable}[l]{p{60pt} p{350pt}}
	$\langle \cdot \rangle$ & Spatial average over the ASI \\
	$\langle \cdot \cdot \rangle$ & Correlation over the ASI \\
	$\odot$ & Hadamard product (element-wise matrix multiplication) \\
	$*$ & Convolution \\
	% $\overline{a}$& time average\\
	%$\av{a b}$& correlation\\
	%$\sav{a}$&spatial average over computational domain\\
	%$\dot{a}$&time derivative\\
	$K()$ & Complete elliptic integral of the first kind \\
	$\sigma( \cdot )$ & Standard deviation for variable sampled from Gaussian distribution \\
	$\norm{ \cdot }$ & Vector norm\\
	&\\
\end{longtable}

\subsection*{Roman symbols}
\begin{longtable}[l]{p{60pt} p{350pt}}
	$a$ & Semi-major axis ($l/2$) \\
	&\\

	$b$ & Semi-minor axis ($w/2$) \\
	$\vc{B}_\mathrm{ext}$ & External magnetic field \\
	&\\

	$d$ & Effective distance between north and south poles of a magnet in the dumbbell approximation \\
	&\\

	$\vc{e}_z$ & Unit vector along the $z$-axis \\
	$E$ & Total interaction energy \\
	$\EB$ & Energy barrier due to shape anisotropy \\ % TODO: what to do with E_B^c? Make a new global rule that (something)^c is critical? Should change T_c then
	$\EBeff$ & Effective energy barrier \\
	$E_\mathrm{Exch}$ & Exchange energy\\
	$E_\mathrm{MS}$ & Magnetostatic interaction energy \\
	$E_\mathrm{Z}$ & Zeeman energy\\
	$E_\perp$ & Energy of a magnet when pointing \ang{90} counter-clockwise from its current state \\
	$\Delta E$ & Switching energy \\
	&\\

	$\mathcal{I}$ & Geometrical factor used for second-order correction on magnetostatic interaction \\
	&\\

	$J$ & Exchange coupling constant \\
	&\\

	$\vc{K}_i$ & Demagnetisation kernel for a magnet $i$ \\
	$\vc{\mathcal{K}}^{(q)}$ & Demagnetisation kernel for a magnet at site $q$ in the unit cell \\
	$K_u$ & Uniaxial shape anisotropy \\
	&\\

	$l$ & Length of elliptical magnet ($2a$) \\
	$L_x$ & Number of cells in the simulation along the x-direction \\
	$L_y$ & Number of cells in the simulation along the y-direction \\
	&\\

	$M$ & Average magnetisation of ASI \\
	$M_\parallel$ & Average magnetisation of ASI along the direction of the external field $\vc{B}_\mathrm{ext}$ \\
	$\vc{M}$ & Magnetisation \\
	$M_\mathrm{sat}$ & Saturation magnetisation \\
	&\\

	$N$ & Number of magnets in the ASI \\ % TODO: add N_x and N_y?
	$\mathcal{N}_i$ & Nearest neighbours of magnet $i$ \\
	&\\

	$q$ & Magnetic charge \\
	$\qNN$ & Local antiferromagnetic parameter in OOP square ASI \\
	$Q$ & Maximum allowed change in switching probability for two simultaneously sampled magnets \\
	&\\

	$\vc{r}_i$ & Position of magnet $i$ \\
	$r_{ij}$ & Distance between magnets $i$ and $j$ \\
	$\vc{r}_{ij}$ & Distance vector between magnets $i$ and $j$ \\
	$\rmin$ & Minimal allowed distance between simultaneously sampled magnets \\
	$\vc{r_N}_i$ & Position of north monopole of magnet $i$ in the dumbbell model \\
	$\vc{r_S}_i$ & Position of south monopole of magnet $i$ in the dumbbell model \\
	$R_x$ & Number of cells along the $x$-axis in a jittered grid subregion \\
	$R_y$ & Number of cells along the $y$-axis in a jittered grid subregion \\
	&\\

	$s_i$ & State of a magnet ($\pm 1$) along the easy axis $\vc{u}_i$ \\
	$\vc{s}$ & Column vector containing all $s_i$ \\
	$\langle S_i S_{i+1} \rangle$ & Nearest-neighbour correlation \\
	&\\

	$t$ & Time \\
	$\Delta t$ & Random switching time interval of a single magnet \\
	$T$ & Temperature \\
	$T_c$ & Critical temperature (Ising model) \\
	&\\

	$\vc{u}_i$ & Uniaxial anisotropy axis of magnet $i$ \\
	&\\

	$V$ & Volume of a magnet \\
	&\\

	$w$ & Width of elliptical magnet ($2b$) \\
	&\\
\end{longtable}

\subsection*{Greek symbols}
\begin{longtable}[l]{p{60pt} p{350pt}}
	% $\alpha$&Gilbert damping parameter\\
	% &\\

	% $\beta$&$=\frac{KV}{k_\mathrm{B} T}$, anisotropy energy with respect to thermal energy\\
	% &\\

	% $\gamma$& gyromagnetic ratio\\
	% $\gamma_0$& $\mu_0\gamma$\\
	% $\Gamma_{XY}$&coviariance between stochastic variables $X$ and $Y$\\
	% &\\

	$\delta$ & Relative magnetostatic/exchange coupling \\
	&\\

	% $\eta$& viscosity\\
	% &\\

	$\theta$ & In-plane magnetisation angle measured clockwise from the easy axis $\vc{u}$ \\
	&\\

	% $\kappa$& windowing weighting factor\\
	% &\\

	% $\lambda$&index parameter to denote a single event or realisation of a stochastic variable\\
	% &\\


	$\mu_i$ & Size of magnetic moment of magnet $i$ \\
	$\vc{\mu}_i$ & Total magnetic moment vector of magnet $i$ \\
	&\\

	
	$\nu$ & Switching rate \\
	$\nu_0$ & Attempt frequency \\
	&\\

	$\rho$ & Perpendicular magnetisation factor \\
	&\\

	% $\sigma$&standard deviation in the lognormal distribution\\
	% $\sigma_X$&standard deviation of the stochastic variable $X$\\
	% $\sigma_X^2$&second central moment of stochastic variable $X$, variance\\
	% &\\

	% $\tau_0$& $\frac{1}{2\nu_0}$\\
	% $\tau_\mathrm{B}$& Brownian fluctuation time constant\\
	% $\tau_\mathrm{eff}$& effective fluctuation time constant\\
	% $\tau_\mathrm{ex}$& timescale of the experiment\\
	% $\tau_\mathrm{N}$& N\'eel fluctuation time constant\\
	% &\\

	$\phi$ & Angle of the effective field w.r.t. easy axis $\vc{u}$ \\
	&\\

	$\chi$ & Uniformly distributed random variable in $(0, 1]$ \\
	&\\
\end{longtable}
