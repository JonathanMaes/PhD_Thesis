\chapter{Introduction}\label{ch:Introduction}
\glijbaantje{It is important to draw wisdom from many different places.\\If you take it from only one place, it becomes rigid and stale.}{Uncle Iroh}

\section{Reservoir computing}\label{sec:1:RC}\indexlabel{reservoir computing}
\subsection{Motivation}
\subsection{Origin}
\subsubsection{ANN}
\paragraph{Feedforward NN}
\paragraph{Recurrent NN}
\subsubsection{LSM}
\subsubsection{ESN}
\subsubsection{Reservoir computing}
\subsubsection{Multi-reservoir techniques}
\paragraph{Single dynamical node}
\paragraph{Rotating neurons reservoir}
\subsection{Metrics}
\subsubsection{Kernel-quality, generalization-capability, compute quality}
\subsubsection{Task-agnostic metrics: nonlinearity, memory capacity, complexity (, parity check)}
\subsubsection{Attractors etc.} % More test-like than metric-like. NARMA, MG, Lorentz
\paragraph{Mackey-Glass}
\subsubsection{Other} % Memorization, frequency generation, classification
\subsection{Applications or tasks}
\subsubsection{Digital signal processing} % Chaotic time series, speech recognition, TODO
\subsection{Physical RC}
\subsubsection{Why physical systems can be used for RC}
\subsubsection{Physical platforms suitable for RC}
\paragraph{Magnetic} % Rings and ASI
In the context of the \spinengine project, three types of magnetic systems were considered.
These are the magnetic nanorings~\cite{DynamicEmergence_NanomagneticSystem}, focused on by the University of Sheffield, in-plane (IP) artificial spin ice (ASI)~\cite{RC_ASI}, researched by NTNU, and out-of-plane (OOP) ASI~\cite{KUR-24}, the primary interest of ETHZ.
Meanwhile, Ghent University provided simulation support for these magnetic systems.
All three present their own advantages and challenges. \par
The nanoring ensembles show promising RC performance, but did not provide straightforward on-chip input and readout methods as would be desirable for applications.
Using anisotropic magnetoresistance (AMR) only provides a single readout value --- thereby obscuring a lot of the system's dynamics --- and requires the use of lock-in amplifiers~\cite{ArchitecturesNanoringRC,Vidamour2023}.
Readout using ferromagnetic resonance (FMR) can provide a higher-dimensional readout in the form of spin-wave spectra~\cite{swindells2024fingerprinting}, but requires bulky waveguides. \par
The potential for RC in IP ASI has been demonstrated numerically~\cite{RC_ASI}, but the experimental application of input and state readout is not straightforward.
An external field is often used for input, though this is undesirable for on-chip applications; instead, it appears possible to use SOT~\cite{SOT_switching_IP}.
Efficient read-out of IP ASI remains challenging, due to the symmetry of the system coupled with the discontinuous nature of the ASI. \par % TODO END: cite roadmap and include more info
Finally, for the OOP ASI efficient input (SOT) and read-out (AMR) mechanisms had been demonstrated experimentally, but their potential for RC had not.
Therefore, this thesis will fill this gap in knowledge and assess the viability of RC with OOP ASI by making use of numerical simulations.
\paragraph{Electronic}
\paragraph{Water bucket}

\section{Artificial spin ice}\label{sec:1:ASI}\indexlabel{artificial spin ice} % TODO END: decide how exactly to use these labels, which words to label like this. Perhaps a good rule of thumb is to label the list of abbreviations, for easy reference to their first occurrence in the text.
\subsection{Nanomagnet(ism)}
\subsubsection{Physics} % See masterproef for this, IP vs. OOP also
\subsection{I/O for RC}\label{sec:1:ASI_IO}
\subsubsection{Input}
\paragraph{SOT}
\cite{SOT_FM_AFM,SOTswitchingCoPt,SOT_Roadmap,vlasov2022optimal}
\paragraph{STT} % Briefly
\subsubsection{Output}
\paragraph{AHE}
\cite{AHE,AHE_Culcer}
\paragraph{AMR} % Briefly
\paragraph{FMR} % Briefly
\paragraph{SHE} % Briefly
\cite{SHE}
\subsubsection{Imaging} % Imaging can also be seen as an "output", though this should be considered a class of its own.
\paragraph{MFM}
\paragraph{PEEM}

