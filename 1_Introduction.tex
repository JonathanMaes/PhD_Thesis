\chapter{Introduction}\label{ch:Introduction} % TODO END: link to index terms
\glijbaantje{It is important to draw wisdom from many different places.\\If we take it from only one place, it becomes rigid and stale.}{Uncle Iroh}
%\glijbaantje{If a machine is expected to be infallible then it cannot also be intelligent.}{Alan Turing}

%This introduction opens with the discussion of and motivation for the research presented in this thesis.
%We then introduce the physical reservoir computing paradigm and the key metrics used to quantify performance.
%Since we focus on magnetic reservoirs,~\cref{sec:1:Nanomagnetism} describes the fundamentals of nanomagnetism and the physical phenomena that enable efficient input and readout.
%Next, we turn to magnetic reservoirs, in particular artificial spin ice, before concluding with a brief description of simulation techniques suitable for these systems.

\section{Context and motivation}\label{sec:1:Context}
Computers are currently used for a wide variety of applications, like scientific simulations, graphics rendering, processing well-structured data, running the internet\dots\,
Conventional computers --- consisting of transistors arranged into logic gates --- are well suited for such tasks which require exact mathematical operations to be executed.
However, our world provides many challenging tasks that can not be captured by such simple, exact sets of predefined rules.
The archetypical example of this is driving a car, but this extends far into other complex tasks that are often thought of as requiring some degree of ``intelligence'' --- whatever that word may entail exactly. \par
The field of \idx{machine learning} aims to tackle these rather complex tasks using artificial systems, the most prevalent of which nowadays are \xref{artificial neural networks} that take inspiration from the synaptic connections in a biological brain.
Over the years, neural networks have advanced significantly, most recently evidenced by the rise of large language models which have demonstrated a broad societal impact~\cite{ImprovingLanguageGPT,GPT-4}.
However, training these systems to the point where they can perform any of the aforementioned tasks at a decent level requires vast amounts of data to fine-tune their ever increasing number of internal parameters --- accompanied by an unsustainably rising energy cost~\cite{QuantumNeuromorphicOpportunities,BLOOM_CarbonFootprint_176Bparam}.
This stands in stark contrast to biological brains, which are capable of learning patterns based on a far more limited set of data, while consuming orders of magnitude less power than an artificial system of equivalent computational capability~\cite{NeuromorphicSpintronics}.
Furthermore, running such neural networks on conventional hardware is encountering limits: to increase performance, the number of trainable parameters is rapidly being increased, but transistor scaling and power density limitations make it hard for hardware to keep up.
Hence, while advances in training algorithms and the ever increasing wealth of available data create ever more potent machine learning systems, the underlying approach is leading to a bottleneck. \par
Several factors that contribute to this bottleneck are often quoted in the context of neural networks or other machine learning algorithms running on conventional hardware:
\begin{itemize}
	\item Conventional hardware uses transistors to perform logical operations.
	The ever increasing computational demands have driven \xlabel{Moore's law}, which states that the number of transistors on a chip doubles every couple of years.
	However, this scaling law appears untenable as transistors approach nanometre scales.
	Relatedly, \xlabel{Dennard scaling} initially allowed the power density of chips to remain roughly constant during this downscaling, but current leakage has inevitably led to increased power dissipation over the past two decades.
	In an attempt to patch this, multi-core processing has become prevalent, but the high power density still limits the number of cores that can operate simultaneously without overheating. % This results in so-called areas of ``dark silicon'' that are unused at a given instant.
	Thus, new kinds of energy-efficient hardware need to be explored.
	\item Conventional computers are built according to the von Neumann architecture, where computation (CPU) is separated from memory (RAM).
	However, transferring information from memory to the processing unit wastes a lot of time and energy, leading to the infamous \idx{von Neumann bottleneck}~\cite{TaskAdaptivePRC}.
	The separation between memory and processing is not fundamentally required: biological brains in the natural world exhibit no such divide as they use a vast network of interconnected neurons, yet they are nonetheless capable of learning, processing and retaining information.
	The field of \idx{in-memory computing} concerns itself with bridging this divide in artificial systems.
	\item Artificial neural networks, like any other machine learning system implemented on conventional hardware, are abstracted far away from the physical domain~\cite{RC_ASI}.
	Most advancements arise from improving their algorithms, but these have to be run on a particular (e.g., von Neumann) computer architecture made up of circuits that handle information encoded by physical devices like transistors~\cite{RC_SuperconductingElectronics}.
	These many layers of abstraction render this approach inherently inefficient, as it does not directly exploit the physical properties of the constituent materials for computation~\cite{RC_ASI}.
\end{itemize}
These various drawbacks of current implementations have become more noticeable over the past decade, as portrayed in~\cref{fig:1:Ngram}.
This has led to the advent of \idx{neuromorphic computing} --- a field of research at the intersection of biology, computer science and physics --- which takes inspiration from the brain to develop more efficient hardware for processing complex data in parallel.
While artificial neural networks are indeed inspired by the brain, their implementation on conventional hardware is typically not considered part of neuromorphic computing, though approaches to implement neural networks on more efficient unconventional hardware are. % rather than applying algorithms on existing transistor-based hardware.
Note that neuromorphic computing does not seek to replace conventional computing: the latter will always find applications in systems that require absolute precision. % think of physical simulations, the transmission of data, or something as common as rendering an image on a screen.
Neuromorphic computing, on the other hand, does not seek exact solutions; in the words of Alan Turing, ``if a machine is expected to be infallible then it cannot also be intelligent.''
%This paradigm shift opens the door to innovative approaches for tackling computational challenges, particularly in areas like real-time sensing, pattern recognition, and prediction.
%Noise can explore a high dimensional landscape related to cost function by escaping local minima.

\vspace{-1.5em}
\sidefig[0.6]{1_Introduction/Ngrams.pdf}{
	\label{fig:1:Ngram}
	Popularity of various key concepts from the year 2000 onwards.
	The surge in popularity of alternative brain-inspired and physical computing methods coincides with raising concerns related to the usage of conventional hardware.
	Data provided by the \href{https://books.google.com/ngrams/graph?content=Neuromorphic computing,Reservoir computing,In-memory computing,von Neumann bottleneck,Dennard scaling,Material computation&year_start=2000&year_end=2022&corpus=en&smoothing=3&case_insensitive=false}{Google Books Ngram viewer} throughout the English corpus, smoothed over a 3-year period.
	\newline
}
\vspace{-1em}

A related concept is \xlabel{material computation}~\cite{NeglectedPillar}, which goes one step further: its goal is to find ways to exploit the inherent computing power present in physical substrates.
Whereas classical computing uses a top-down approach where mathematics is imposed onto a physical material --- for instance to construct logic gates --- material computation works in the opposite direction by instead asking what computations a physical object performs naturally~\cite{RC_ASI}.
By designing materials with specific behaviours, systems can be created that naturally exhibit the desired computational properties, reducing the levels of abstraction that currently exist between physics and computation. \par
Among the various substrates being explored for material computation, magnetic and spintronic (spin transport electronic~\cite{Spintronics}) hardware are especially appealing~\cite{grollier2020neuromorphic,NeuromorphicSpintronicsProspect,QuantumNeuromorphicOpportunities}.
Single-domain ferromagnetic nanomagnets, for example, have the ability to store information, respond non-linearly to stimuli~\cite{NeuromorphicSpintronics} and can dissipate very little energy during information processing~\cite{ThermodynamicLimitsComputation,SpintronicsEnergyEfficientComputing}.
The \link{magnetostatic interaction}{magnetostatic (MS) coupling} within ensembles of nanomagnets provides further advantages, as it can give rise to rich collective behaviour like phase transitions, which the constituent magnets would not exhibit in isolation~\cite{NeuromorphicSpintronicsProspect,RC_ASI}.
This inspired the European \spinengine project --- within the context of which the work presented in this thesis was performed --- to research how tunable ensembles of nanomagnets could be used for material computation. \par % three types of magnetic reservoirs were considered: magnetic nanorings~\cite{DynamicEmergence_NanomagneticSystem}, in-plane ASI~\cite{RC_ASI}, and perpendicular-anisotropy ASI~\cite{KUR-24}.
A well-established example of such an ensemble is \xref{artificial spin ice} (ASI): an ordered lattice of bistable $\approx \SI{100}{\nano\metre}$-size nanomagnets where the lattice geometry does not allow all local interactions to be simultaneously satisfied, leading to emergent dynamics~\cite{ASIpyrochlores}.
The other aforementioned factors have made them a platform of particular interest for scalable and energy-efficient material computation~\cite{PhD_Stromberg}.
ASI with out-of-plane magnetisation offers particularly convenient electrical interfacing for both input and readout, but its computational capability had yet to be demonstrated. \par % In comparison to in-plane ASI, for which the computational capability had already been demonstrated numerically~\cite{ASI_computation,hon2021numerical}
This provided the main motivation for this thesis, ``\!\emph{\phdtitle}''.
A more detailed description of ASI follows in~\cref{sec:1:ASI}; first,~\cref{sec:1:RC} introduces the general principles of \xref{reservoir computing}.

\newpage
\section{Reservoir computing}\label{sec:1:RC}
% Introduction to RC can take inspiration from: https://arxiv.org/html/2412.13212v1
At the intersection of neuromorphic computing and material computation lies the paradigm of \idx{reservoir computing}.
To explain the underlying concept, we must first take a step back and have a closer look at the aforementioned artificial neural networks.

\subsection{Origin}
\paragraph{Artificial neural networks}
The most ubiquitous machine learning models currently in use revolve around \idx{artificial neural networks} (ANNs).
These were originally inspired by the brain: an ANN consists of a network of non-linear summing nodes (akin to neurons) connected by directed weighted edges (akin to synapses)~\cite{EvaluatingRestrictedESNs,zurada1992introduction,AIsimulateMemoryContinuity}.
Each node holds a numerical value: during operation, this value is determined based on the weighted sum of all nodes that feed into it.
This allows activations to propagate through the network~\cite{lukovsevivcius2009reservoir}.
To prevent these values from accumulating without limit, they are typically bounded to the range $[0,1]$ by applying an \xlabel{activation function} to the weighted sum --- typically a sigmoid, like the logistic map $e^x/(1+e^x)$ or hyperbolic tangent~\cite{SurveyUniversalApproximation}.
A node that uses such an activation function is referred to as an \idx{artificial neuron}.
The non-linearity introduced by the activation function is what enables ANNs to approximate any arbitrary function, allowing them to adequately separate, group and otherwise process input data~\cite{ApproximationFNN,SurveyUniversalApproximation,funahashi1992neural}.
For many real-world tasks, the form of the desired function is unknown, so heuristics are used to ``train'' the weights of all edges based on a given dataset~\cite{EvaluatingRestrictedESNs}.
How exactly this training is done, depends on the structure of the ANN.
Typically, a major distinction is made between \xlabel{feed-forward neural network}s (FNNs) and \xlabel{recurrent neural network}s (RNNs)~\cite{ApproximationRNN}. \par
In a FNN, the nodes are grouped in layers, with connections between all the nodes of two successive layers.
This is schematically illustrated in~\crefSubFigRef{fig:1:NeuralNetworks}{a}: from the blue input node(s) on the left, information propagates to the right through the grey weighted connections, into the green hidden layers, finally yielding one or multiple output values at the red node(s)~\cite{zurada1992introduction}.
FNNs are mainly used for static data processing, such as image classification\footnote{Visual tasks are typically performed using convolutional neural networks --- a particular type of FNN that extracts features from images.}~\cite{ApproximationRNN}.
While~\crefSubFigRef{fig:1:NeuralNetworks}{a} shows a small network of 9 nodes inside 2 hidden layers, many real-world tasks require far larger networks with millions or even billions of weights~\cite{ImageNet,BLOOM_CarbonFootprint_176Bparam,GPT-J-6B}. \par
RNNs are conceptually similar to FNNs, with the important distinction that their connection topology possesses cycles, either in a structured or unstructured manner~\cite{lukovsevivcius2009reservoir,Hopfield1982}.
The existence of cycles has a profound impact, as an RNN may develop self-sustained temporal activation dynamics along its recurrent connection pathways, even in the absence of input.
Hence, mathematically speaking, RNNs are dynamical systems while FNNs are functions~\cite{lukovsevivcius2009reservoir}. % REF verbatim
This enables an RNN to preserve a non-linear transformation of the input history in its internal state, providing dynamical memory and making RNNs especially well-suited for dynamic data processing~\cite{RC_RecentAdvances,ApproximationRNN}.
However, whereas FNNs can be trained using methods such as \xlabel{backpropagation}~\cite{Backpropagation} or gradient descent, the recurrent connections in RNNs make training very computationally expensive and difficult~\cite{DifficultyTrainingRNN,EvaluatingRestrictedESNs,Moon_2021,RC_SuperconductingElectronics,funahashi1992neural}.
Some approaches to train RNNs take inspiration from FNN: back-propagation through time (BPTT), for instance, unfolds the recurrent connections in time to obtain an FNN on which backpropagation can be applied~\cite{jaeger2002tutorial}.
However, this is not so straightforward, leading to issues regarding the calculation of the required gradients and a tendency to get stuck in local minima~\cite{RC_Tensegrity,DifficultyTrainingRNN,D-ESN-Improved}. % Other methods are real-time recurrent learning (RTRL) and extended Kalman filtering (EKE), both explained in jaeger2002tutorial.

\vspace{-1em}
\xfig{1_Introduction/NeuralNetworks.pdf}{
	\label{fig:1:NeuralNetworks}
	Schematic representation of \textbf{(a)} a feed-forward neural network (FNN), \textbf{(b)} a recurrent neural network (RNN) and \textbf{(c)} reservoir computing (RC).
	In the latter, the reservoir is treated as a black box; it can be of mathematical origin, like a RNN, or a physical system like artificial spin ice (ASI).
	Grey connections propagate information from left to right, unless indicated by an arrow.
	Blue (red) nodes represent input (output), while green nodes constitute the hidden layers of the network.
}
\vspace{-1em}

\paragraph{Echo state networks and liquid state machines}
To address the shortcomings of conventional RNN training methods like BPTT, two alternative concepts were independently proposed in the early 2000s, laying the foundation for what would eventually be known as reservoir computing.
Jaeger~\cite{jaeger2001echo} presented echo state networks (ESNs), while Maass~\etal~\cite{maass_LSM} presented liquid state machines (LSMs)\footnote{
	Their names reveal some fundamental desirable properties that the random recurrent network should provide.
	In LSMs, the sparse network of spiking neurons projects inputs into a high-dimensional transient state, akin to waves in a liquid.
	The ``echo'' of ESNs refers to the fading memory provided by the internal dynamics of a recurrent network, where loops can retain information from past inputs.
}.
Both are recurrent networks and operate on very similar principles: their main difference lies in their nodes, as LSMs use \xlabel{spiking neurons} while nodes in ESNs use sigmoidal activation functions.
The crucial property that sets them apart from other training methods is that their underlying recurrent network is randomly initialised and remains unmodified during training~\cite{ReviewESNs,RC_Tensegrity}.
Instead, only a single linear readout layer is trained via linear regression to obtain the desired output from this random network~\cite{D-LSM,D-ESN-Improved}.
This decoupling of internal dynamics from learning is the core insight that underpins both LSMs and ESNs.
Training ESNs and LSMs is a walk in the park as compared to BPTT, as linear regression is far less computationally demanding and does not require the (sometimes problematic) calculation of gradients~\cite{D-ESN-Improved}. \par
From these principles, it is but a small step to reservoir computing (RC), which generalises these ideas by treating the internal network as a black box~\cite{RC_unification,D-ESN-Improved,RC_Tensegrity}.
In the next section we will explore the reservoir computing framework in more detail.

\subsection{Reservoir computing}
Reservoir computing is a machine learning framework suitable for temporal processing tasks~\cite{BookReservoirComputing}.
The concept revolves around a ``reservoir'' --- an unspecified non-linear dynamical system --- that projects the input data into a higher-dimensional space, thereby facilitating the separation of said data by a linear transformation~\cite{appeltant2011information,KUR-24,RC_ASI}. % TODO: use figure like Fig. 2 in ~\cite{appeltant2011information}
The principle is schematically illustrated in \crefSubFigRef{fig:1:NeuralNetworks}{c}.
A weighted sum of the components of (a lower-dimensional representation of) the reservoir's state vector (green nodes), then provides the output (red node) of the system as a whole. % REFs for RC, specifically training a readout layer: Dale et al., 2017; Jensen and Tufte, 2017; Sillin et al., 2013
These weights can be trained with a simple learning algorithm like linear regression in order to produce a desired output~\cite{RC_RecentAdvances, RC_SuperconductingElectronics}.
Crucially, \textit{the properties of the reservoir itself are not modified during training} --- only the final linear readout layer is~\cite{RC_ASI,DynamicEmergence_NanomagneticSystem}. \par
Since the reservoir acts as a black box, its properties can not be changed to obtain the desired response\footnote{
	It is possible to adjust a physical reservoir to exhibit the desired properties for a given task~\cite{AdaptiveProgrammableRC,gartside2022reconfigurable}, but this is different from mathematically tuning the weights of the readout layer via a simple procedure like linear regression.
}.
To solve temporal tasks, a reservoir should be a high-dimensional non-linear system with short-term memory~\cite{NeuromorphicAFMspintronics,RC_RecentAdvances}.
Reservoirs are therefore not limited to being purely mathematical concepts, as many physical systems naturally possess these properties~\cite{RC_DipoleNanomagnets,RC_PassiveFrustratedNM,RC_ASI,RC_RecentAdvances,NeuromorphicOscillators,VowelRecognition4STO,RC_DiffusiveMemristors,RC_MemristorTemporal,gartside2022reconfigurable}, ranging from a bucket of water~\cite{PatternRecognition_Bucket} to photonic systems~\cite{RC_Photonic} and superconducting electronics~\cite{RC_SuperconductingElectronics} --- see~\cref{sec:1:PhysicalRCplatforms}.
This is the concept of \idx{physical reservoir computing} (PRC)~\cite{PRC}, which allows the usage of physical systems without further mathematical abstractions --- apart from the linear readout layer at the end~\cite{RC_RecentAdvances}. \par
While this avoids the many layers of abstraction present in conventional machine learning methods, physical constraints can make it difficult to optimise a physical reservoir~\cite{RC_RecentAdvances}, in contrast to RNNs which have full mathematical freedom.
For instance, when using a physical reservoir to perform a temporal task, the physics of the system are most effectively harnessed when the timescale of system dynamics matches that of incoming data patterns~\cite{KUR-24}, but the timescales of physical phenomena are often hard to control.
Since physical reservoirs can often be hard to scale, multiple techniques have been devised to improve performance by combining multiple reservoirs together~\cite{EvaluatingRestrictedESNs,RotatingNeuronsRC} or by time multiplexing~\cite{appeltant2011information}.
Nonetheless, PRC combines many advantages of all the previously discussed concepts: not much data is required for training, which is computationally very cheap due to the single linear readout layer, enabling low energy usage due to the direct usage of a physical substrate for computation.
The combination of these factors has attracted significant interest to (P)RC.

\paragraph{Reservoir requirements}
For a physical substrate to be suitable as a reservoir, several requirements need to be satisfied: high dimensionality, non-linearity, fading memory, and a separation/approximation property~\cite{appeltant2011information}.
High dimensionality and non-linearity cooperate, as a non-linear mapping of inputs from a small-dimensional into a high-dimensional space can enable the separation of otherwise linearly inseparable inputs, making them both essential properties for classification tasks~\cite{VoltageControlled_SuperparamagneticRC,RC_ASI,appeltant2011information,RC_RecentAdvances}. % The dimensionality is related to the number of independent signals obtained from the reservoir~\cite{RC_RecentAdvances}.
In prediction tasks, the non-linearity of a reservoir can enable the extraction of non-linear dependencies from the inputs for better inference. \par
Fading memory~\cite{boyd1985ApproximatingVolterra} originates from the echo state property of ESNs, where recurrent connections preserve information of the past while it slowly fades out as new input values are applied.
This property ensures that the reservoir state is dependent on recent inputs, while being independent of inputs in the distant past, which is useful behaviour for most temporal processing tasks~\cite{ChaoticTimeSeries_ML,appeltant2011information}. \par % If inputs from the distant past are somehow needed, they can be incorporated using "long short-term memory" (Hochreiter and Schmidhuber).
Finally, the \xlabel{separation property} states that distinct signals should yield distinct responses, while the \xlabel{approximation property} requires similar inputs to remain grouped together in the output space, making the system insensitive to noise~\cite{RCbenchmarksReview1}.
The former benefits from a chaotic regime, while the latter (and fading memory) are more easily fulfilled in a stable regime~\cite{RC_RecentAdvances}.
As such, many physical reservoirs aim to balance on this ``edge of stability'' whenever it is available in the system, to balance between these properties~\cite{appeltant2011information}.

\paragraph{Mathematical description}
% TODO: add symbols to the figure
In the reservoir computing paradigm, an input signal $s(t)$ perturbs a reservoir $\mathcal{R}$ --- which can be any physical or abstract non-linear dynamical system --- eliciting a response $\mathcal{R}[s](t)$.
The full internal state of the reservoir following this perturbation is often unknown, so a $p$-dimensional readout $\vc{x}(t) \in \mathbb{R}^p$ is assumed (i.e., the green nodes in~\cref{fig:1:NeuralNetworks}).
The final output value\footnote{
	A one-dimensional output is assumed here, but this is not a fundamental restriction: multiple sets of weights can be trained independently if multiple outputs are desired.
} $\hat{y}(t)$ of the system is then obtained by a weighted sum $\hat{y}(t) = \vc{x}^\mathrm{T}(t) \cdot \vc{w}$ of the components of the readout vector $\vc{x}(t)$.
Depending on the task performed, one or multiple such sums can be trained and interpreted in the context of that specific task~\cite{RC_RecentAdvances}. \par % For example, temporal pattern classification may provide multiple outputs, while prediction and generation only uses a single one
When the desired input-output relation $s(t) \mapsto y(t)$ is known for a limited set of data --- the training set --- the weights $\vc{w}$ can be adjusted to minimise the difference between the output $\hat{y}(t)$ and the desired output $y(t)$.
This is usually done by \xlabel{ordinary least-squares} (OLS) regression, which requires only a single matrix inversion.
By sampling the readout and desired output at $n$ moments in time, $n$ vectors $\vc{x}_i$ and $\vc{y}_i$, $i = 1, \dots, n$ are obtained.
These can be arranged into an $n \times p$ readout matrix $\vc{X}$ (whose row $i$ is $\vc{x}_i^\mathrm{T}$) and the $n$-dimensional desired output vector $\vc{y}$.
The optimal weights can then directly be calculated as
\begin{equation}
	\vc{w} = (\vc{X}^\mathrm{T} \vc{X})^{-1} \vc{X}^\mathrm{T} \vc{y} \mathrm{,}
\end{equation}
which minimises $\norm{\vc{y} - \hat{\vc{y}}}^2 = \norm{\vc{y} - \vc{X}\vc{w}}^2$~\cite{RC_NNN}.
These optimal weights then remain fixed, and the performance of the reservoir can be evaluated by feeding it the test set --- a previously unknown, yet similar data set as the training set.

\subsection{Metrics} \label{sec:1:RC_metrics}
To compare different physical reservoirs or identify optimal system parameters, metrics are needed that can provide a consistent evaluation of how suitable a given physical system is to be used as a reservoir.
In this respect, two types of metrics are often used: those calculated based on the ranks of matrices, and those calculated through linear estimators.

\subsubsection{Matrix rank-based metrics}\indexlabel{matrix rank-based metrics} \label{sec:1:RC_metrics_KQ}
Legenstein and Maass~\cite{WhatMakesPowerful} proposed three metrics, based on the readout vector $\vc{x}(t)$ of the reservoir.
By periodically sampling this $p$-dimensional vector at $n$ moments in time, an $n \times p$ matrix $\vc{X}$ can be constructed whose rows are these vectors at different moments in time.
Depending on the choice of input sequence, the rank of this matrix gives information about certain properties of the system~\cite{RC_ASI}.

\paragraph{Kernel-quality}
The \idx{kernel-quality} $K$ measures the \xref{separation property}, i.e. how well the reservoir can separate distinct temporal input patterns.
Practically, it is determined by applying $n$ random input sequences and recording the readout after every sequence, each of which consists of a large number of discrete input values.
Here, we will use $n=p$ to obtain a square matrix for a more consistent measure.
Using random sequences ensures that they are maximally distinct, such that the rank of this matrix (i.e., the kernel-quality $K$) is a measure for how distinct the responses obtained from the reservoir can be~\cite{Vidamour_2022}.
A more efficient alternative method would be to record the response $\vc{x}$ after each input value, as this would also result in maximally distinct input sequences, though this method strays further from the original definition~\cite{RC_HierarchicalNeuroevolution,RCbenchmarksReview1}.
Hence, a high kernel-quality $K$ indicates that the reservoir exhibits the separation property.
In the extreme case where $K=n$, any desired output can be implemented by the reservoir in combination with its linear readout layer~\cite{WhatMakesPowerful}.

\paragraph{Generalisation-capability}
The \xref{kernel-quality} is insufficient to quantify the computational performance of a reservoir~\cite{WhatMakesPowerful,RC_ASI,IL_Masterproef}.
The \idx{generalisation-capability} $G$ is a similar metric, which measures the \xref{approximation property} i.e., whether similar input sequences yield similar responses.
Practically, it is determined in the same way as the kernel-quality $K$, except that the input sequences are now similar in one way or another.
This can be done by adding a small amount of noise to a given input sequence, but we will follow the method outlined in~\cite{RC_ASI} instead: the first \SI{60}{\percent} of each sequence will be identical among all of them, while the remaining \SI{40}{\percent} is random for each sequence individually.
This way, the rank of the resulting matrix (i.e., the generalisation-capability $G$) measures how distinct the responses of similar inputs are.
A low value of $G$ is therefore desirable, as it indicates robustness to noise and overfitting, as the reservoir adheres to the approximation property~\cite{RCbenchmarksReview1}.

\paragraph{Computing capacity}
To obtain a single metric for the computational capability of a reservoir, Legenstein and Maass proposed to simply subtract the normalised values of both these metrics.
This yields the \idx{computing capacity} $C = K - G$, where higher values typically indicate better performance.
However, care should be taken with such claims, as high computing capacity does not necessarily mean that the reservoir will perform well on all tasks it is given.

\subsubsection{Linear estimator-based metrics}
A practical definition of non-linearity and memory capacity was given by Love~\etal~\cite{RC_TaskAgnosticMetrics_v2}, which will also be used here.
The parity check metric uses a similar calculation that is better suited for binary input sequences~\cite{hon2021numerical,tsunegi2019STOforcedsyncRC}.

\paragraph{Non-linearity}
The non-linearity of the reservoir can be measured by considering that its response consists of a linear and a non-linear contribution.
The linear contribution can be extracted by training a linear estimator $\hat{\vc{x}}(t)$ for the current reservoir readout based on the past input $s(t - \tau)$.
More specifically, the estimators
\begin{equation}
	\hat{x}_i(t) = c_i + \sum_{\tau=0}^{k} w_{i,\tau} s(t - \tau)
\end{equation}
are trained by adjusting the weights $w_{i,\tau}$ and $c_i$ based on the known discrete-time input-output relationship (on the training set).
The cut-off time $k$ must be longer than the relaxation time of the reservoir.
By measuring the quality of this estimate using the $\mathrm{R}^2$ correlation coefficient, a measure of the linearity of each separate readout node is obtained.
This can be converted to an overall measure for non-linearity as
\begin{equation}
	\mathrm{NL} = 1 - \frac{1}{p} \sum_{i=1}^p \mathrm{R}^2[\hat{x}_i;x_i] = 1 - \frac{1}{p} \sum_{i=1}^p \frac{\mathrm{cov}^2(\hat{x}_i; x_i)}{\sigma^2(\hat{x}_i) \sigma^2(x_i)} \mathrm{,}
\end{equation}
which represents the mean non-linearity of all readout nodes as a value $\mathrm{NL} \in [0,1]$.
Since a value of 0 indicates a perfectly linear response, higher values of NL are more desirable.

\paragraph{Linear memory capacity}
The linear memory capacity can also be calculated through by means of linear estimators, but by working the other way around: an estimator for the past input $s(t - \tau)$ is now trained based on the current readout $\vc{x}(t)$ of the reservoir~\cite{tsunegi2019STOforcedsyncRC,NeuromorphicFewShot}.
In other words, the estimator
\begin{equation}
	\hat{s}(t - \tau) = c_\tau + \sum_{i=1}^{p} w_{i,\tau} x_i(t)
\end{equation}
is now trained.
The linear memory capacity is then obtained by summing the $\mathrm{R}^2$ correlation coefficient of all estimators up to a cut-off time $k$:
\begin{equation}
	\label{eq:1:MC}
	\mathrm{LMC} = \sum_{\tau = 1}^{k} \mathrm{R}^2[\hat{s}(t - \tau); s(t - \tau)] = \sum_{\tau = 1}^{k} \frac{\mathrm{cov}^2(\hat{s}_i; r_i)}{\sigma^2(\hat{s}_i) \sigma^2(s_i)} \mathrm{,}
\end{equation}
which lies in the range $[0,k]$ and represents the approximate number of previous input values that the reservoir can recall.
Note that the memory capacity could be increased arbitrarily high by choosing a high cut-off time $k$.
However, the memory capacity typically exhibits a plateau beyond a certain value of $k$, before rising dramatically only when $k$ approaches the size of the training set.
Hence, $k$ should be chosen at this plateau. \par
Note that this measures short-term memory; long-term memory often refers to the knowledge engrained in the weights of an ANN i.e., the linear readout layer in the case of RC.
Furthermore, this only captures the linear memory capacity: the non-linear memory capacity is far more computationally expensive as it requires a summation over polynomials of different degrees~\cite{RCbenchmarksReview1}.
We therefore limit this discussion to the linear memory capacity.

\paragraph{Parity check}
Closely related to the memory capacity is the parity check metric~\cite{hon2021numerical,tsunegi2019STOforcedsyncRC}.
This is particularly intended for use with binary input, as it defines the parity of a binary input sequence $s(t)$ as
\begin{equation}
	\pi_\tau(t) = \Bigg[\sum_{j = 0}^{\tau} s(t - j) \Bigg] \mathrm{mod}~2 \mathrm{.}
\end{equation}
The parity check (PC) is then defined in the same way as the memory capacity in~\cref{eq:1:MC}, except with $\pi_\tau(t)$ instead of $s(t - \tau)$.
This introduces a form of non-linearity into it, though it still mostly serves as a measure of the memory capacity in binary systems.

\subsubsection{Benchmarks} % Memorisation, frequency generation, classification
In addition to these six metrics, reservoirs are often used to perform specific benchmark tasks in literature.
Example of these include signal transformation and prediction~\cite{AdaptiveProgrammableRC,jaeger2001echo,JaegerHaasWireless,RC_MemristorTemporal,Sunspots_Shougat,gartside2022reconfigurable,appeltant2011information,Vidamour2023}, or speech and image recognition~\cite{farronato2022reservoir,grollier2020neuromorphic,DynamicEmergence_NanomagneticSystem,Vidamour2023}. % See https://arxiv.org/html/2405.06561v
Wringer~\etal~\cite{RCbenchmarksReview1} provide a comprehensive overview of various benchmarks used throughout literature.
They note that --- while less general than metrics --- benchmarks enable a more concrete comparison of reservoirs, whereas metrics provide a more abstract notion of computational capability.

\subsection{Physical reservoir platforms}\label{sec:1:PhysicalRCplatforms}
The first true example of physical reservoir computing came in the form of a bucket of water: this proof-of-concept provided input through actuators, and the resulting wave patterns were successfully used for speech recognition~\cite{PatternRecognition_Bucket}.
Similarly, gas bubbles in water have served as an effective reservoir~\cite{RC_GasBubbles}.
While these examples highlight the universal nature of reservoir computing, such macroscopic systems are not very practical.
To allow on-chip integration and achieve low energy consumption, a wide variety of microscopic systems have been exploited as reservoirs~\cite{RC_RecentAdvances}. \par
Traditional machine learning algorithms, like FNN and RNN, are often implemented on traditional transistor-based hardware.
However, for RC specifically, simpler electronic devices exist that can potentially save energy and enable faster computation~\cite{RC_RecentAdvances}.
Memristors are a well-established example: they can store information in their resistance by changing their internal ion distribution, allowing them to act analogously to biological neurons~\cite{MemristorArtificialNeuron,MemristiveNN,RC_DiffusiveMemristors,RC_MemristorTemporal}.
Networks of nanowires are also interesting for PRC, as they can self-assemble into a complex recurrent network with memristive junctions forming at intersections~\cite{RC_NNN}.
However, memristors are rather slow.
%A faster reservoir can be made of field-programmable gate arrays (FPGAs), which are especially suitable for integer-valued tasks due to their digital nature~\cite{RC_RapidTimeSeries,BookReservoirComputing}. % Autonomous Boolean logic
%For instance, FPGAs can efficiently implement cellular automata reservoirs, like elementary rule 90 that exhibits chaotic behaviour and can be implemented solely using XOR gates~\cite{RC_CA}.
Superconducting materials, on the other hand, offer low dissipation and extremely fast operation ($\gg \SI{}{\giga\hertz}$), with a number of possible architectures, like for instance a Josephson transmission line~\cite{RC_SuperconductingElectronics}. \par % See also: analogue electronic circuits [TaskAdaptivePRC, Ref. 10], ferroelectrics [TaskAdaptivePRC, Ref. 13], VLSI
Photonic principles have also been exploited to construct reservoirs with very fast operation and low power consumption~\cite{RC_RecentAdvances}.
They make use of on-chip structures like optical waveguides, amplifiers and delay lines~\cite{RC_RecentAdvances}.
One option is to use a single non-linear node, and use delay lines to perform time-multiplexing~\cite{appeltant2011information,RC_AllOptical}.
Another option is to use arrays with non-linear components and delay lines to introduce phase shifts between interconnections, enabling computation through the interference of light~\cite{RC_Photonic,RC_PhotonicSi}. \par
Mechanical systems can also be used as reservoirs, but are not as fast or efficient as the aforementioned systems.
However, mechanical systems provide an extreme example of physical reservoir computing, as it has been shown that the body of a tensegrity robot (tensile structures arranged in trusses, consisting of both springs and incompressible bars) can be used for both locomotion and computation~\cite{RC_Tensegrity}.
Back to the microscopic scale, vibrational modes in micro-electronic mechanical systems (MEMS) can provide the required non-linearity that a reservoir needs.
It is also possible to use chemical processes for computation, for example using processes inspired by those in living cells~\cite{NanoscaleRC,ElectrochemicalPRC,RC_Chemical}. \par
Finally, a wide variety of magnetic and spintronic systems have been proposed for use as physical reservoirs.
Since this thesis focuses on nanomagnetic systems, it is warranted to first explain the physical principles of nanomagnetism before continuing this discussion in~\cref{sec:1_RC_magnetic}.

\newpage
\section{Nanomagnetism} \label{sec:1:Nanomagnetism}
\subsection{Physics of nanomagnets}
This thesis focuses on the usage of magnetic systems as reservoirs.
Generally, magnetism and magnetic fields originate from the movement of electric charges.
A current flowing in a loop --- be it a conducting wire or the orbital motion of electrons in an atom --- will experience a torque $\vc{\tau} = \vc{\mu} \times \vc{B}$ when placed in a magnetic field $\vc{B}$~\cite{IntroMagneticMaterials}.
This defines the so-called \idx{magnetic moment} $\vc{\mu}$ associated with this circulating current.
Conversely, an object with a magnetic moment also generates its own magnetic field
\begin{equation}
	\vc{B}(\vc{r}) = \frac{\mu_0}{4 \pi} \ab[\frac{3 \vc{r} (\vc{\mu} \cdot \vc{r})}{r^5} - \frac{\vc{\mu}}{r^3}] \mathrm{,}
\end{equation}
giving rise to the \xref{magnetostatic interaction} through which magnetic moments influence each other. % (\cref{eq:2:E_MS})
Freely rotatable magnetic moments will therefore prefer a parallel alignment along their connecting axis.

\paragraph{Ferromagnetism}
Some materials, like a bar magnet, exhibit a spontaneous \idx{magnetisation}: they generate a magnetic field seemingly without any electrical current.
This magnetisation $\vc{M}$ --- the magnetic moment per unit volume --- arises at the atomic level from two distinct sources: the angular momentum of electrons around the nucleus, and the intrinsic spin of the elementary particles comprising an atom~\cite{coey2010magnetism}.
The latter is often dominant, and while the magnetic moments of pairs of electrons often cancel out, atoms with unpaired electrons can exhibit a net magnetic moment~\cite{PhD_Leliaert}. % See slide 35 of https://magnetism.eu/esm/2019/slides/simonet-slides1(magnetism_of_atoms).pdf
However, this does not yet mean that a material composed of such atoms will exhibit a macroscopic magnetic moment like a bar magnet, as the moments of nearby atoms are not necessarily aligned. \par
A non-vanishing net magnetic moment is the privilege of \textit{ferromagnetic}\indexlabel{ferromagnetism} materials.
It is only in these materials that the \idx{Heisenberg-Dirac exchange interaction}~\cite{heisenberg1928theorie} provides a strong tendency for nearby atomic magnetic moments to align, leading to a non-zero macroscopic magnetisation within the material.
This alignment is only possible below a material's \idx{Curie temperature}: at room temperature, \ce{Fe}, \ce{Ni} and \ce{Co} are the only pure elements to be ferromagnetic, though many compounds like \ce{Nd2Fe14B} and \ce{SmCo5} are as well.

\paragraph{Ferromagnetic domains}
The keyword regarding the exchange interaction is ``nearby'': over longer distances, the magnetisation direction can vary throughout the material as other interactions are also at play.
Most importantly, the magnetostatic interaction makes it energetically favourable for the magnetisation to curl back on itself. % Minimising stray fields
This competition between the exchange and magnetostatic interaction gives rise to $\SI{}{\micro\metre}$-sized \idx{ferromagnetic domains} with near-uniform magnetisation, separated by domain walls.
These are the reason why a macroscopic piece of ferromagnetic material often does not exhibit a spontaneous net magnetic moment, as all these domains cancel out when no external field is present~\cite{coey2010magnetism}. \par
The magnetisation can thus be described as a vector field $\vc{M}(\vc{r})$ with a resolution of $\lesssim \SI{5}{\nano\metre}$, as the exchange interaction dominates over shorter length scales.
This forms the basis of micromagnetic theory~\cite{mumax3}: over such small distances, the local magnetisation always has approximately the same magnitude; $\norm{\vc{M}(\vc{r})} \approx M_\mathrm{sat}$, the \idx{saturation magnetisation}.
An applied magnetic field does not affect $M_\mathrm{sat}$, as it is an intrinsic property that determines the strongest magnetic moment per volume that a given material can provide.
Instead, a field only affects the magnetisation direction, causing some domains to grow or shrink.

\paragraph{Single-domain nanomagnets}
In sufficiently small thin-film ferromagnetic particles (with lateral dimensions $\lesssim 100$ to \SI{500}{\nano\metre}), the magnetostatic interaction is too weak for separate domains to form~\cite{Kittel_TheoryFMDomains,BrownThermalFluctuations}.
This results in a \idx{single-domain nanomagnet} with a nearly uniform magnetisation, giving it a magnetic moment $\vc{\mu}$ close to the upper limit of $M_\mathrm{sat}V$~\cite{FRENKEL1930,neel1949theorie}. % Frenkel and Dorfman first conceived the idea of a single-domain particle based on energy considerations.
When such a nanomagnet exhibits some form of anisotropy, a preferential magnetisation axis will emerge, referred to as the \idx[nolabel]{easy axis}~\cite{nisoli2013colloquium}.
This has led to nanomagnets being used for memory applications, as the two magnetisation directions along the easy axis can be related to bits `0' and `1'~\cite{MQCA_RoomTemp,NML_Carlton,Gypens_Balanced,Gypens_SelfOrganizing,JM_Masterproef}.
The magnetisation can switch between these two stable states under the influence of nearby nanomagnets, external magnetic fields, thermal fluctuations, electrical currents and a variety of other phenomena~\cite{SwitchingForced_EnergyEfficient,BrownThermalFluctuations,neel1949theorie}. \par
One common source of anisotropy is the geometry of a nanomagnet: this \idx[nolabel]{shape anisotropy} arises from the interaction of the magnetisation with its own demagnetising field, resulting in a low-energy state when the magnetisation aligns with the longest axis of the geometry~\cite{MagneticCharge}.
Anisotropy can also originate from the crystal structure, but this can be suppressed by an appropriate choice of material; for example, permalloy (\ce{Fe20Ni80}) is an alloy specifically designed to exhibit minimal \xlabel{magnetocrystalline anisotropy}.
Finally, \xlabel{interfacial anisotropy} can occur between the ferromagnetic material and a substrate.
For example, at the interface between \ce{Co} and \ce{Pt}~\cite{PMA_PdCo_PtCo}, a \idx[nolabel]{perpendicular magnetic anisotropy} (PMA) exists that acts in the immediate vicinity of the interface, where the magnetisation then prefers to align perpendicular to said interface.

\paragraph{In-plane and out-of-plane nanomagnets} % Q: is this OK? What to do with the chapter 3 intro? Just remove 3.1.1?
Two types of nanomagnets are typically distinguished: in-plane (IP) and out-of-plane (OOP) nanomagnets.
IP nanomagnets can readily be obtained by tailoring the geometry of the ferromagnetic particle: a flat geometry guarantees an in-plane magnetisation direction through the \xref{shape anisotropy}.
This easy plane can be reduced to an \xref{easy axis} by using an elongated two-dimensional shape, like an ellipse or rectangle.
OOP magnets are harder to realise using only shape anisotropy, as it is often infeasible to create tall structures on wafers.
This requires them to use other types of anisotropy to favour a magnetisation direction perpendicular to the substrate, like the aforementioned PMA between specific materials like \ce{Co} and \ce{Pt}.
The structure of OOP nanomagnets will be discussed in more detail in~\cref{sec:3:OOP_nanomagnet_PMA}~and~\ref{sec:3:E_contributions}.

\subsection{Input and readout for reservoir computing}\label{sec:1:ASI_IO}
To use nanomagnetic systems for RC, methods are required to inject information and read out their resulting magnetic states.
One of the main factors that makes magnetic systems attractive candidates for RC is their ease of electrical interfacing, as many effects can manifest between currents and magnetic materials, enabling efficient input and output.
In particular, for OOP systems --- the main focus of this thesis --- \xref{spin--orbit torque} provides an efficient input method that synergises well with readout via the \xref{anomalous Hall effect}.

\subsubsection{Input}
In practice, using magnetic fields to switch nanomagnets is not ideal.
External coils are frequency-limited by their inductance, while on-chip striplines can provide more localised fields but remain relatively inefficient.
Furthermore, if the magnetic fields are not properly confined to the targeted nanomagnets, they may inadvertently disturb other parts of the system.
In contrast, all-electrical methods exist that bypass these issues by directly injecting spins into the magnet, thereby exerting a torque on the magnetisation.
This offers much faster, more energy-efficient control without requiring strong magnetic fields, making them far better suited for on-chip implementations.

\paragraph{Spin-transfer torque}
A well-established method for switching nanomagnets is \idx{spin-transfer torque} (STT)~\cite{SlonczewskiSTT}, as has been widely used in e.g., magnetic random-access memory (STT-MRAM)~\cite{brataas2012current}.
When an electrical current with a spin polarisation $\vc{\sigma}$ enters a ferromagnetic material, the injected spins will exert a torque $\vc{\tau}_\mathrm{STT} \propto \vc{\mu} \times (\vc{\mu} \times \vc{\sigma})$ on the local magnetisation $\vc{\mu}$, causing it to align with the injected spin polarisation.
Typically, this polarisation is generated by passing a current through a stack of two ferromagnetic layers: the free layer (i.e., the thin-film nanomagnet to be switched) and a fixed layer whose magnetisation is known or controlled by other means.
When both layers are sufficiently close together for the spin polarisation to survive between them, the STT will promote a known magnetisation direction in the free layer~\cite{SOT_FM_AFM,mumax3tutorial}. % remain coherent?
In OOP nanomagnets, both layers are therefore often stacked on top of each other, with the current passing vertically through both of them.
While this is an effective method, such a multilayer structure increases fabrication complexity and requires the magnet stack to be fully conductive (or at least allow tunnelling), which can be limiting in some situations.

\paragraph{Spin--orbit torque}
More recently, switching through \idx{spin--orbit torque} (SOT) has attracted attention as an alternative to STT, as it eliminates the need for a second magnetic layer or vertically flowing current~\cite{mumax3tutorial}.
Instead, it only requires a heavy-metal layer with strong spin--orbit coupling (e.g., \ce{Pt}, \ce{Ta}, \ce{W} \dots) to be adjacent to the ferromagnet~\cite{SOT_Roadmap}.
Typically, a heterostructure is used where a ferromagnet is patterned on top of a heavy-metal underlayer --- for instance, the OOP magnets considered in~\cref{ch:Applications} are \ce{Co}-\ce{Pt} multilayers~\cite{KUR-24}. % exactly the type of structure that supports SOT.
When an in-plane electrical current passes through the underlayer, a spin polarisation will accumulate at the interface with the ferromagnet as a result of the bulk spin Hall effect~\cite{SHE} and/or interfacial Rashba-Edelstein effect~\cite{SOT_Rashba}. % RE effect due to symmetry being broken at an interface. The bulk spin Hall effect often provides the dominant damping-like torque, while any interfacial Rashba effect contributes mainly to the field-like term. % SOT does not originate from a single underlying phenomenon, but is an umbrella term for various effects that all relate to spin--orbit coupling and end up generating a torque in a magnetic material.
These spins, polarised in-plane and perpendicular to the electrical current, can then diffuse into the ferromagnet and exert a torque on its magnetisation~\cite{mumax3tutorial}.
Note that this is indeed distinctly different from STT, where the spin polarisation originates from another magnet and flows straight through the structure~\cite{SOT_Roadmap}. \par
This torque is often decomposed into two orthogonal components.
The field-like torque $\vc{\tau}_\mathrm{FL} \propto \vc{\mu} \times \vc{\sigma}$ causes a precession of the magnetic moment $\vc{\mu}$ around the spin polarisation $\vc{\sigma}$.
On the other hand, the damping-like component $\vc{\tau}_\mathrm{DL} \propto \vc{\mu} \times (\vc{\mu} \times \vc{\sigma})$ aligns the magnetic moment with the spin polarisation~\cite{SOT_Roadmap,SOT_FM_AFM}.
The latter is functionally similar to STT, earning it the name Slonczewski-like SOT.
It has been shown that the field-like SOT can act as a strong transverse effective field, upwards of $\SI{1}{\tesla}$ per $\SI{e8}{\ampere\per\centi\metre\squared}$~\cite{SOT_Rashba}, making it a very efficient method to switch OOP nanomagnets~\cite{vlasov2022optimal,SOTswitchingCoPt,SHE_CurrentInducedSwitching,SpintronicsEnergyEfficientComputing}. \par
However, SOT alone does not guarantee deterministic switching of an OOP ferromagnetic layer, as the in-plane spin polarisation does not differentiate between the `up' and `down' magnetic states.
To resolve this, additional symmetry breaking is required, more specifically with respect to the plane formed by the electrical current and the interface normal~\cite{SOT_Roadmap}.
This can be achieved in a plethora of ways e.g., by an in-plane magnetic field, an exchange bias layer, a source of OOP spin polarisation, exploitation of crystal symmetries if possible, or geometrical asymmetry in the system~\cite{SOT_firstprinciplesCoPt,SOT_Roadmap}.
The polarity of switching depends on the direction of the applied current, the material, and the symmetry breaking.
Hence, when it is known which current direction corresponds to a preferential `up' or `down' magnetisation, electrical control of the magnetisation can be achieved~\cite{SOT_PMAinsulator}.

\subsubsection{Output}
\paragraph{Anomalous Hall effect}
Our method of choice to obtain electrical output from ensembles of out-of-plane nanomagnets is to make use of the \idx{anomalous Hall effect} (AHE).
In the ordinary Hall effect, an electrical current flowing through a conductor (the Hall bar) generates a transverse voltage, since the current generates a magnetic field which exerts a Lorentz force on the electrons.
When the conductor is a ferromagnet, an additional contribution appears that is proportional to the magnetisation of the material.
While its exact origin is still a matter of debate, it is thought to originate from spin--orbit coupling: spin-polarised electrons passing through the ferromagnet get scattered, leading to a charge imbalance across the width of the conductor~\cite{AHE_Culcer,AHE}. \par
Rather than express it as a voltage, the Hall effect is more often described by the transverse resistivity (i.e., the ratio between transverse electric field $E_x$ and longitudinal current density $J_y$) $\rho_{xy} = E_x/J_y = R_0 H_z + R_\mathrm{AHE} M_z$, with the Hall bar in the $xy$-plane~\cite{SHE,AHE}. % TODO: should these symbols be in the symbols list? Especially H vs. B
The first term captures the ordinary Hall effect, the latter the AHE with $M_z$ the OOP magnetisation component and $R_\mathrm{AHE}$ the anomalous Hall coefficient~\cite{AHE}.
Hence, the AHE is greatest when the magnetisation is perpendicular to the conductor plane and has opposite sign for opposite magnetisation directions --- good news for the OOP nanomagnets that we focus on.
This has enabled the usage of AHE for readout in many SOT-MRAM devices, as it can directly be applied in the same heavy-metal/ferromagnet heterostructure as SOT. \par
The readout resolution depends on various factors like $R_\mathrm{AHE}$ and the precise geometry of the Hall bar, with single-magnet resolution achievable. % Typically ~100nm
Since the AHE synergises well with the geometry of OOP nanomagnetic ensembles, it is the method of choice to obtain readout from such reservoirs.

\paragraph{Anisotropic magnetoresistance}
Another phenomenon that can provide readout is \idx{anisotropic magnetoresistance} (AMR): when an electrical current flows through a ferromagnetic material, the resistance weakly depends on the angle $\theta$ between this current and the magnetisation~\cite{AMR}.
More specifically, the total resistance $R(\theta) = R_\perp + (R_\parallel - R_\perp) \cos(2\theta)$~\cite{Hu2023}.
This makes it possible to distinguish between perpendicular and parallel magnetisation states.
While this is a very accessible resistance measurement-based technique, it has several limitations. \par
First off, while permalloy exhibits a relatively strong AMR, the absolute effect remains rather small in comparison to the total resistance --- at most a few percent.
This can make it hard to accurately determine the magnetisation state without the use of lock-in amplifiers~\cite{ArchitecturesNanoringRC,Vidamour2023}.
Furthermore, due to the $\cos(2 \theta)$-dependence, AMR does not reveal information about the magnetisation direction, only about the local angle between the magnetisation and current.
It also requires the system to be electrically connected: this makes it an appropriate choice for e.g. the ring system in the \spinengine project~\cite{Vidamour2023,ArchitecturesNanoringRC,DynamicEmergence_NanomagneticSystem}, but prevents it from directly being applied to ensembles of nanomagnets.
In interconnected ferromagnets, it only provides a limited amount of information for RC as a lot of the system's dynamics are obscured.
Ensembles of OOP nanomagnets, in particular, would need the current to flow vertically through each magnet, further complicating the system.

\paragraph{Ferromagnetic resonance}
Another possible method to obtain rich output from a reservoir is \idx{ferromagnetic resonance} (FMR)~\cite{AdaptiveProgrammableRC,gartside2022reconfigurable}.
This involves applying a microwave field and measuring the absorption as a function of frequency.
When the frequency matches a resonance of the internal magnetisation dynamics of a nanomagnet, an absorption peak appears.
The complexity of internal magnetisation dynamics, represented in the frequency domain, can provide a rich and non-linear reservoir output~\cite{AdaptiveProgrammableRC,Gomez-Iriarte_FMR}.
However, the peripheral equipment required to perform the technique renders it energy-inefficient and unsuitable for on-chip applications.
Furthermore, as the wavelengths involved in FMR are on the order of centimetres, it does not provide local information --- though for RC purposes it does not need to, since the output space is rich regardless.

\subsubsection{Imaging}
Aside from electrical measurements that provide information about the state of (an ensemble of) magnets, it is also possible to directly image the magnetic configuration of a system through microscopy techniques~\cite{freeman2001advances}.
While these can also be seen as ``output'', they are distinct from the aforementioned electrical measurements as they can not be integrated on-chip.

\paragraph{Magnetic force microscopy}
A relatively common technique is \idx{magnetic force microscopy} (MFM)~\cite{MFM}, which scans a cantilever with a very sharp tip across the sample, similar to atomic force microscopy.
Whereas the latter uses a non-magnetic tip to measure the sample's texture at an atomic scale, MFM uses a tip coated in ferromagnetic material (e.g., \ce{Co}), giving it a vertical magnetic moment.
This enables a measurement of the out-of-plane\footnote{
	It is also possible to measure the in-plane magnetisation by using an in-plane magnetised tip, but this complicates the interpretation as this mixes two components of the magnetisation~\cite{MFM_inplane}.
} magnetic field by scanning the tip at a very low height ($\approx \SI{20}{\nano\metre}$) above the sample~\cite{NML_Carlton,JM_Masterproef}.
However, this does not only measure the magnetic field, but also the atomic forces that a non-magnetised tip would experience.
Therefore, a standard atomic force microscopy scan is first performed to obtain the texture profile of the sample, such that a subsequent MFM measurement can keep the magnetic tip at a constant height above the sample, thereby isolating the magnetic field strength from the atomic forces. \par
The result is a greyscale image, where the brightness indicates the vertical component $\partial B_z/ \partial z$ of the magnetic field gradient close to the sample.
For example, an IP nanomagnet will appear as a pair of regions with bright and dark contrast, because the divergence (convergence) of the magnetic field at its north (south) pole results in opposite vertical magnetic fields on either end~\cite{NML_Carlton}.
An MFM image of OOP magnets (\cref{sec:3:MFM}) is easier to interpret, as there is a one-to-one correspondence between $B_z$ and the the magnetisation state. \par
MFM is relatively cheap to set up as compared to other magnetic microscopy techniques.
It can achieve high resolution, typically on the order of $\lesssim \SI{100}{\nano\metre}$, depending on the height and size of the tip~\cite{MFM}.
However, it is very slow; scanning across a $\SI{}{\micro\metre}$-sized sample can take multiple minutes.
Care must also be taken that the sample is not affected by the magnetic moment of the tip, though this influence is often negligible as the tip is made very small~\cite{Probing_MagnetoOptics}.

\paragraph{Other magnetic microscopy techniques}
While MFM is the most commonly available magnetic microscopy technique, X-ray and optical methods are also often used~\cite{DynamicEmergence_NanomagneticSystem}.
These can provide different information about nanomagnetic systems, as they are directly sensitive to the magnetisation rather than stray fields, making them easier to interpret in systems with in-plane magnetisation.
X-ray based measurements like photo-electron emission microscopy (PEEM)~\cite{PEEM} and scanning transmission X-ray microscopy (STXM)~\cite{Imaging_MTXM} can make use of the \xlabel{X-ray magnetic circular dichroism} (XMCD) effect to achieve even higher resolution than MFM.
However, they require circularly polarised X-rays, which are typically only available at synchrotron radiation facilities.
Another option is to use optical techniques like the magneto-optical Kerr effect (MOKE)~\cite{KerrFaraday_book}: while these do not require a synchrotron, their resolution is diffraction-limited to several hundred \SI{}{\nano\metre}. \par
Recently, scanning nitrogen-vacancy (NV) microscopy has emerged as a high-sensitivity scanning-probe technique.
Like MFM, it uses a cantilever, but with a diamond tip in which one or more NV colour centres are embedded~\cite{NVprospects}.
This enables a precise measurement of the \textit{vector} magnetic field through the shift it induces in the energy levels of the NV centres. % vector field (both strength and direction) --> simplifies interpretation in complex geometries
This shift can be detected through a decrease in fluorescence of the colour centre when driven by a microwave field that matches the NV resonance~\cite{NVprinciples_QDM}.
Hence, NV microscopy does not disturb the sample since it is an all-optical technique, with high resolution achievable ($\approx \SI{10}{\nano\metre}$) due to the small size of the NV centre. % REF: qnami white paper

\subsection{Magnetic platforms for reservoir computing}\label{sec:1_RC_magnetic}
Let us now return to our discussion of physical reservoir platforms from~\cref{sec:1:PhysicalRCplatforms}, where we previously left off at magnetic and spintronic reservoirs.
Such systems are alluring due to their non-linearity and non-volatility (i.e., memory), as well as low-power electrical interfacing through the various methods outlined in the preceding section.

\subsubsection{Overview of spintronic reservoirs}
Spin-torque nano-oscillators (STOs) are a common example of devices considered as spintronic reservoirs, owing to their small size and high frequency operation~\cite{tsunegi2019STOforcedsyncRC}.
They are, in essence, a magnetic tunnel junction (MTJ): a stack of two ferromagnetic layers, one fixed and one free, separated by an insulator.
When a current tunnels through the stack, the magnetisation of the free layer will rotate due to STT, at a microwave frequency that depends non-linearly on the input current.
This way, by applying time-multiplexed input, a single STO has successfully been used~\cite{tsunegi2019STOforcedsyncRC} as a single dynamical node to, for instance, perform speech recognition~\cite{STO_RC_Riou2021,NeuromorphicOscillators}.
Arrays of STOs are also more broadly used in neuromorphic computing to implement traditional neural networks more efficiently in hardware~\cite{VowelRecognition4STO}. \par
The magnetisation within ferromagnetic thin films can also exhibit a diverse range of magnetic textures, like domain walls, spin waves and skyrmions.
Domain walls~\cite{Venkat_2024} can get stochastically (de)pinned at defects and can mutually interact, which can result in a dynamic equilibrium that is suitable for RC~\cite{DynamicEmergence_NanomagneticSystem}.
They can even exhibit inertia, and the position of a single domain wall can already encode sufficient information for RC~\cite{RC_DW}.
Spin waves excited in ferromagnets with low damping (e.g., YIG) may provide complex interference patterns that can also be used in RC~\cite{RC_SpinWaveInterference}, and can also interact with domain walls.
Skyrmions are non-trivial topological bubble-like features that typically exist at cryogenic temperatures.
Their position and structure can be affected by currents and spin waves, which the skyrmions can in turn affect back, giving rise to a plethora of options to obtain the required non-linearity and fading memory~\cite{RC_TaskAgnosticMetrics_v2,RC_SkyrmionCrystalSW}. \par
This thesis concerns itself with ensembles of magnetostatically coupled nanomagnets.
Due to competing magnetostatic interactions, these ensembles often exhibit collective dynamics like phase transitions and long-range order, which can be exploited for reservoir computing.
A disordered arrangement of nanomagnets has, for example, been used to predict boolean functions like those produced by cellular automata~\cite{RC_PassiveFrustratedNM,RC_DipoleNanomagnets}. % OOP
A class of ensemble of particular interest is artificial spin ice~\cite{RC_ASI}.

\subsubsection{Artificial spin ice}\label{sec:1:ASI}
\textit{Artificial spin ice}\indexlabel{artificial spin ice} (ASI) is a type of metamaterial where magnetostatically coupled single-domain (bistable) nanomagnets are arranged on an ordered lattice~\cite{RC_ASI,flatspin}.
The magnetisation direction of the constituent nanomagnets can switch between two states under the influence of nearby magnets, externally applied magnetic fields, thermal fluctuations, or any of the electronic input methods previously discussed~\cite{CoerciveFieldReversal,BrownThermalFluctuations,SOT_FM_AFM,brataas2012current}. \par
A defining property of ASI is \idx{frustration}, meaning that the underlying lattice does not allow all local pairwise interactions to be simultaneously satisfied.
The term ``spin ice'' draws inspiration from water ice, where the oxygen atoms are positioned on a diamond lattice, so each has four nearest neighbours with which it shares a hydrogen bond. % TODO FIG: figure of water ice, then spin ice, then ASI?
However, each oxygen atom is closely bound to two hydrogen atoms, so the hydrogen atoms in the two other bonds must be more distant, leading to geometrical frustration with a ``two-in, two-out'' ice rule~\cite{nisoli2013colloquium,ZeroPointEntropy,heyderman2013artificial,MagnetizationDynamicsASI}.
Similar to this, natural spin ices have been found in certain crystal structures like pyrochlores, whose atomic magnetic moments must obey similar ice rules that also lead to frustrated spin configurations~\cite{nisoli2013colloquium}.
From there, it is only a small leap to ASI, where a 2D lattice of nanomagnets mimics the frustration present in natural spin ices~\cite{heyderman2013artificial}.
Frustration can lead to correlations and collective behaviour that the individual elements would not exhibit by themselves, resulting in non-trivial dynamics and complex magnetic orderings~\cite{AdvancesASI,ASI_computation,ApparentFMpinwheel}. \par
ASI is very attractive in comparison to its natural spin ice counterpart.
Modern nano-fabrication methods such as electron beam lithography allow any arbitrary geometry to be realised and many other aspects of the system to be engineered, offering enormous freedom to the designer~\cite{AdvancesASI,ASI_computation}.
Furthermore, the mesoscopic size of ASI (a few hundred nanometers) enables direct observation of its magnetic degrees of freedom through a variety of microscopy techniques and electrical readout methods, in contrast to natural spin ices where imaging individual spins is not feasible~\cite{nisoli2013colloquium,freeman2001advances}.
The aforementioned electrical input methods also enable precise control over these systems.
Hence, the controlled, tunable and easily measurable environment offered by ASI has enabled the study of complex phenomena such as phase transitions with ordered domains~\cite{ApparentFMpinwheel,sklenar2019field,MeltingASI,ImagingBridgedKagome,sendetskyi2019continuous,lou2023competing,branford2012emerging} or glass-like behaviour~\cite{wang2006artificial,ZeroPointEntropy}, vertex-based frustration~\cite{morrison2013unhappy,nisoli2018topologytetris} leading to emergent topological structures such as monopoles and Dirac strings~\cite{ObservationMonopoleASI,mengotti2011kagome}, chiral dynamics~\cite{branford2012emerging,EmergentChiralityRatchet}...
ASI also holds significant potential for computational applications~\cite{heyderman2022spin}, both in conventional logic~\cite{ComputationalLogic_2018,Gypens_Balanced,EngineeringRelaxationComputation} and neuromorphic computing~\cite{ASI_computation,RC_RecentAdvances}. \par
% Q: remove the following paragraph or keep it?
%The study of ASI has since broadened well beyond the scope of frustrated systems, and now the term is used more broadly for ensembles of nanomagnets exhibiting a wide range of physical phenomena~\cite{ASIPathsForward}.
%A vast wealth of both periodic and aperiodic arrangements of nanomagnets has been explored throughout the literature, whose behaviour can vary wildly~\cite{nisoli2013colloquium,heyderman2013artificial,AdvancesASI,ASI_Evolutionary_ALife}.
%For example, a superferromagnetic alignment is most favourable in pinwheel ASI (\crefSubFigRef{fig:2:ASIs}{a,b}), while square-lattice ASI (\crefSubFigRef{fig:2:ASIs}{c,d,i}) prefers an antiferromagnetic (AFM) alignment~\cite{ApparentFMpinwheel}.
%All three present their own advantages and challenges. \par
%Nanoring ensembles show very promising RC performance, but do not provide straightforward on-chip input and readout methods.
%For input, rotating external fields were typically used.
%AMR readout obscures a lot of the system's dynamics and requires the use of lock-in amplifiers, while FMR can provide a higher-dimensional readout in the form of spin-wave spectra~\cite{swindells2024fingerprinting} but requires bulky waveguides. \par % Only provides a single readout value, pretty much
The reservoir computing potential of IP ASI has been demonstrated numerically~\cite{RC_ASI}, but the experimental application of input and state readout is not straightforward.
For OOP ASI, efficient input (SOT) and readout (AMR) mechanisms are available, but its RC potential has not yet been demonstrated.
Therefore, the motivation behind the work presented in this thesis was to fill this knowledge gap, by assessing the viability of reservoir computing with OOP ASI.

\newpage
\section{Modelling}\label{sec:1:Modelling}
To determine whether an ASI possesses desirable properties for it to be used as a reservoir, the impact of various system parameters and input methods on reservoir performance must be determined.
Since an experimental exploration of this vast design space would take prohibitively long, a suitable simulation framework is needed.

\subsection{Micromagnetic simulations}
The internal magnetisation dynamics of ferromagnetic structures are the result of the interactions between millions of atomic magnetic moments.
This results in an N-body problem that often has no analytical solution, apart from a select few cases~\cite{PhD_Abert}.
Hence, virtually all practical situations can only be solved in a numerical manner. \par
One particularly useful approximation for the simulation of nanomagnets is \idx{micromagnetic theory}.
It starts from the previously noted observation that, at the scale of the \xlabel{exchange length} $\lambda \sim \SI{5}{\nano\metre}$, the magnetisation within ferromagnets is approximately continuous.
Hence, micromagnetic theory describes it as a continuous vector field $\vc{M}(\vc{r}) = M_\mathrm{sat} \vc{m}$ with constant norm ($\norm{\vc{m}} = 1$).
Such a continuum approximation is valuable as it enables the simulation of $\SI{}{\micro\metre}$-size systems in a reasonable amount of time through a finite-difference or finite-element discretisation with cells of size $\leq \lambda$.
Examples of finite-difference micromagnetic codes include OOMMF~\cite{OOMMF} and the GPU-accelerated \mumax~\cite{mumax3} and \mumaxplus~\cite{MOR-24}.
Finite-element codes are also available, like Nmag~\cite{Nmag} and magnum.fe~\cite{magnumFE}. \par % Other codes: MagnumNP~\cite{magnumNP} (FD), Boris (FD), tetmag (FE), FeeLLGood (FE)
The main complication in micromagnetic simulations is the infinite range of the magnetostatic interaction, which must therefore be calculated between all cells in the simulation.
This can pose an issue, since the high spatial resolution needed ($\leq \lambda$) can lead to simulations that contain millions of cells.
Furthermore, the time step must be tiny, as magnetisation dynamics can occur over fractions of picoseconds~\cite{PhD_Leliaert}.
Therefore, since ASI may contain hundreds of magnets whose switching rate is often orders of magnitude slower, micromagnetic simulations quickly become infeasible~\cite{leo2021chiral}.
Furthermore, to determine statistically valid RC metrics, many input cycles must be applied --- on the order of 100 for the linear estimator-based metrics --- during which many switches occur~\cite{RC_TaskAgnosticMetrics_v2}.
To address these limitations, specialised ASI simulation tools have been developed.

\subsection{Macrospin approximation}\label{sec:1:Modelling_macrospin}
Since the magnetisation of single-domain nanomagnets is often nearly uniform throughout, a \idx{macrospin approximation} can be used where the magnet is represented as a point dipole.
When a nanomagnet has a single easy axis --- as is typically the case in ASI --- it can be approximated as a single Ising spin.
Using such higher-level approximations greatly reduces the computational cost of simulations.
This enables the study of collective behaviour in much larger systems and over far longer timescales, as these simulations are limited by the number of switches rather than the elapsed time.
However, this approach comes at the cost that the internal magnetisation structure of individual nanomagnets is no longer simulated in detail. \par
One example using such a macrospin approximation is the high-performance \texttt{flatspin} simulator~\cite{flatspin} --- also used in the \spinengine project --- which implements deterministic spin flipping via a generalised \xlabel{Stoner-Wohlfarth model}~\cite{StonerWohlfarth2008}.
This model revolves around the effective field experienced by each nanomagnet, whose switching is determined by an angle-dependent \xlabel{coercive field} --- it is easier to switch a nanomagnet by applying a field at an oblique angle w.r.t. the easy axis, rather than parallel or perpendicular to it.
These fields are calculated based on a point dipole approximation.
This results in deterministic dynamics, unless temperature effects are included through a random effective thermal field. \par
Besides deterministic simulations, Monte Carlo methods are also often used to simulate both IP~\cite{Qi2008,Cugliandolo2017,LocalizedFrustratedKagome,Brunn2021,Farhan2013,ApparentFMpinwheel} and OOP~\cite{Chioar2014,PerpendicularMagnetizationASI} ASI. % Qi: kagome NN model, Cugliandolo: square vertex-based model, LocalizedFrustratedKagome: kagome NN model, Brunn: IP square 10NN model, ApparnetFMpinwheel: full MS model, Chioar: square magnetostatic, PerpendicularMagnetizationASI: model G and Z.
However, these are typically specialized to a select few lattice geometries.
Early works often accounted only for nearest-neighbour (NN) interactions, whose strength was either arbitrarily set or calculated separately using micromagnetic codes~\cite{Qi2008,PerpendicularMagnetizationASI}. % REFS: 'moller2006artificial': bit broader using dipolar interaction, though seemingly still only for nearest neighbors. 'lou2023competing': dipole model for half-occupation IP Ising. 'sendetskyi2019continuous': dipole model for square ASI. 'EngineeringRelaxationComputation': KMC on dipole model, seemingly for square arrays.
More recently, the importance of long-range magnetostatic interactions has been emphasized to improve correspondence to experiments~\cite{Chioar2014,Rougemaille2011,Brunn2021}.
As such, vertex~\cite{gilbert2014emergent,Saglam2022Tetris,Goryca2021Plasma,MeltingASI} or hybrid NN/vertex charge~\cite{Canals2016,zhang2013crystallites} models have become more prevalent, especially for square-lattice ASI. % 'MeltingASI': 16-vertex ice model, 'gilbert2014emergent': MC sims based on a vertex model and interacting magnetic charges. 'zhang2013crystallites': uses monopoles and NN couplings to model kagome ASI with Metropolis and loop update. 
Beyond-NN interaction models, including full magnetostatically coupled models~\cite{ApparentFMpinwheel,mengotti2011kagome}, are also used for e.g. square-~\cite{Brunn2021,Farhan2013,sklenar2019field} and triangular-based~\cite{Chioar2014,Rougemaille2011,Hofhuis2020} lattices. \par % 'sklenar2019field': 3NN interactions calculated by mumax on quadrupole lattice. 'mengotti2011kagome' use a full dipole model with Ewald summation for PBC.
Here, we opt to develop our own simulation toolkit: \hotspice~\cite{MAES-24}.
This was driven by a desire to blend the approach of \texttt{flatspin}~\cite{flatspin} --- a general-purpose GPU-accelerated IP ASI simulator --- with the rather purpose-built Monte Carlo toolkits listed above, to obtain efficient simulation of RC in OOP ASI.
The result is a versatile Monte Carlo simulator that aims to capture IP and OOP ASI physics with minimal arbitrary parameters (most importantly, the magnets' anisotropy, magnetic moment and temperature) for a variety of lattice configurations.
System dynamics are determined by the switching probability of each magnet, which is calculated based on the energy barrier separating its two stable states.
Given the importance of considering long-range \xref{magnetostatic interactions} --- particularly for OOP systems --- as highlighted by Chioar~\etal~\cite{Chioar2014}, \hotspice{} explicitly accounts for the magnetostatic interactions between all magnets.
Various optional model variants have also been implemented to enable a closer match to experiments.
The details of the \hotspice simulator will be extensively discussed in~\cref{ch:Hotspice}, after which it is used in~\cref{ch:Applications} to explore reservoir computing in OOP ASI.
