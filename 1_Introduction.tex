\chapter{Introduction}\label{ch:Introduction}
\glijbaantje{It is important to draw wisdom from many different places.\\If you take it from only one place, it becomes rigid and stale.}{Uncle Iroh}

\section{Reservoir computing}\label{sec:1:RC}\indexlabel{reservoir computing}
\subsection{Motivation}
\subsection{Origin}
\subsubsection{ANN}
\paragraph{Feedforward NN}
\paragraph{Recurrent NN}
\subsubsection{LSM}
\subsubsection{ESN}
\subsubsection{Reservoir computing}
\subsubsection{Multi-reservoir techniques}
\paragraph{Single dynamical node}
\paragraph{Rotating neurons reservoir}
\subsection{Metrics}
\subsubsection{Kernel-quality, generalization-capability, compute quality}
\subsubsection{Task-agnostic metrics: nonlinearity, memory capacity, complexity (, parity check)}
\subsubsection{Attractors etc.} % More test-like than metric-like. NARMA, MG, Lorentz
\paragraph{Mackey-Glass}
\subsubsection{Other} % Memorization, frequency generation, classification
\subsection{Applications or tasks}
\subsubsection{Digital signal processing} % Chaotic time series, speech recognition, TODO
\subsection{Physical RC}
\subsubsection{Why physical systems can be used for RC}
\subsubsection{Physical platforms suitable for RC}
\paragraph{Magnetic} % Rings and ASI
\paragraph{Electronic}
\paragraph{Water bucket}

\section{Artificial spin ice}\label{sec:1:ASI}\indexlabel{artificial spin ice} % TODO END: decide how exactly to use these labels, which words to label like this. Perhaps a good rule of thumb is to label the list of abbreviations, for easy reference to their first occurrence in the text.
\subsection{Nanomagnet(ism)}
\subsubsection{Physics} % See masterproef for this, IP vs. OOP also
\subsection{I/O for RC}
\subsubsection{Input}
\paragraph{SOT}
\paragraph{STT} % Briefly
\subsubsection{Output}
\paragraph{AHE}
\paragraph{AMR} % Briefly
\paragraph{FMR} % Briefly
\paragraph{SHE} % Briefly
\cite{SHE}
\subsubsection{Imaging} % Imaging can also be seen as an "output", though this should be considered a class of its own.
\paragraph{MFM}
\paragraph{PEEM}

\section{Simulations}\label{sec:1:sim}
\subsection{(Markov chain) Monte Carlo algorithms}
\subsubsection{Ergodicity and detailed balance}
\subsubsection{Glauber vs. Metropolis}
