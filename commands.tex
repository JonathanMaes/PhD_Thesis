% \definecolor{figcolor}{HTML}{F2F2FF}
\definecolor{figcolor}{HTML}{F2F2F2}
\definecolor{ugentblue}{RGB}{10,30,96}

\newcommand{\colorcode}{(Color code on page \pageref{colorcode}.)\xspace}

\newcommand{\jonathanskip}{\smallskip}
\newcommand{\idx}[1]{\emph{#1}\index{#1}\label{#1}\label{#1s}} 					%%no xspace
\newcommand{\indexlabel}[1]{\index{#1}\label{#1}}
\newcommand{\xlabel}[1]{#1\index{#1}\label{#1}}

\newcommand{\numberthis}{\addtocounter{equation}{1}\tag{\theequation}} % Numbers a single line in a no-numbering multiline equation* or align*

%\newcommand{\debug}[1]{}
%\newcommand{\debug}[1]{{\tiny{(\pageref{#1})}}\normalsize}

\newcommand{\xref}[1]{\hyperref[#1]{#1}\xspace}
\newcommand{\xrefs}[2]{\hyperref[#2]{#1#2}\xspace} %% meervoud etc, add {s}
\newcommand{\link}[2]{\hyperref[#1]{#2}\xspace}
\newcommand{\crefSubFigRef}[2]{\crefformat{figure}{Fig.~##2##1{#2}##3}%
  \cref{#1}\crefformat{figure}{fig.~##2##1##3}}
\newcommand{\ccite}[1]{Ref.~\cite{#1}} % Ref. [1] instead of [1]
\newcommand{\Py}{\xref{permalloy}}
%%vector
% \newcommand{\vc}[1]{\textbf{#1}}
\newcommand{\vc}[1]{\boldsymbol{\mathbf{#1}}} % Use this instead of \vb, which is deprecated anyway alongside the "physics" package
\newcommand{\circled}[1]{\raisebox{.5pt}{\textcircled{\raisebox{-.9pt} {#1}}}}
\newcommand{\sub}[1]{\ensuremath{_{\mathrm{#1}}}}
\newcommand{\super}[1]{\ensuremath{^{\mathrm{#1}}}}
\newcommand{\av}[1]{\langle #1 \rangle}
\newcommand{\sav}[1]{\langle\hspace{-0.4mm}\langle #1 \rangle\hspace{-0.4mm}\rangle}
%\newcommand{\sav}[1]{\left<\hspace{-0.4mm}\left< #1 \right>\hspace{-0.4mm}\right>}
\newcommand{\tpar}[1]{\frac{\partial #1}{\partial t}}%partial derivative to time
\newcommand{\tder}[1]{\frac{\mathrm{d} #1}{\mathrm{dt}}}%full derivative to time

\newcommand{\EB}{E_\mathrm{B}}
\newcommand{\EBtilde}{\widetilde{\EB}}
\newcommand{\kB}{\ifmmode k_\mathrm{B} \else $k_\mathrm{B}$ \fi}
\newcommand{\kBT}{\ifmmode k_\mathrm{B}T \else $k_\mathrm{B}T$ \fi}

% \newcommand{\abs}[1]{\ensuremath{\left| #1 \right|}\xspace}
\newcommand{\diff}[2]{\frac{\partial #1}{\partial #2}}			    %% derivative
\newcommand{\dispdiff}[2]{{\partial #1}/{\partial #2}}			    %% derivative
\newcommand{\difff}[2]{\frac{\partial^2 #1}{\partial #2^2}}			%% second derivative
\newcommand{\ddt}[1]{\ensuremath{\partial #1 / \partial t}}
%\newcommand{\d}{\ensuremath{\mathrm{d}}}
\newcommand{\ddiff}[2]{\frac{\d #1}{\d #2}}
\newcommand{\lapl}{\ensuremath{{\Delta}}}
\newcommand{\eqdef}{\stackrel{\small{\mathrm{def}}}{=}}
\newcommand{\inprod}{\cdot}
\newcommand{\bcdot}{\boldsymbol{\cdot}}
\renewcommand{\quote}[1]{``#1''}
\newcommand{\etal}{\textit{et al.}\xspace}

\newcommand{\code}[1]{\textsf{#1}}
\newcommand{\shell}[1]{{\texttt{#1}}\smallskip\\}
\newcommand{\python}[1]{\lstinline{#1}}

\newcommand{\hide}[1]{}

\newcommand{\glijbaantje}[2]{\vspace{-0.5cm}\begin{flushright}{\textit{#1}\\--- #2\\}\vspace{0.5cm}\end{flushright}}
\newcommand{\publicatie}[1]{Material from this chapter has also been published in \cite{#1}.}

\newcommand{\mumax}{mumax\textsuperscript{3}\xspace}
\newcommand{\mumaxplus}{mumax\textsuperscript{+}\xspace}
\newcommand{\oommf}{\textsc{oommf}\xspace}
\newcommand{\nmag}{Nmag\xspace}
\newcommand{\hotspice}{\textsc{Hotspice}\xspace}

\newcommand{\eqn}[1]{equation \ref{#1}\xspace}
\newcommand{\secateur}[1]{section \ref{#1} (page \pageref{#1})\xspace}
\newcommand{\figuur}[1]{figure \ref{#1} (page \pageref{#1})\xspace}
\newcommand{\Figuur}[1]{Figure \ref{#1} (page \pageref{#1})\xspace}


\newcommand{\kolommen}[2]{
	\begin{minipage}{0.445\textwidth}
		#1
	\end{minipage}
	\begin{minipage}{0.445\textwidth}
		#2
	\end{minipage}
	\smallskip
}

\newcommand{\smallfig}[3]{
	\smallskip
	\begin{center}
		\includegraphics[width=#1\linewidth]{#2}\footnotesize{\\\textsf{(#3)}}
	\end{center}
	\smallskip
}

% 2 args: caption, tabular environment
% 1 optional arg: label
\newcommand{\xtable}[3][nolabel]{
	\jonathanskip\vspace{1em}
	\begin{tikzpicture}
		\node[rounded corners=10pt, fill=figcolor, inner sep=10pt] (figbox) {
			\begin{minipage}{0.955\linewidth}
		  		\centering
		  		\textsf{\tabcaption{#2\label{#1}}}
		  		{\sf
					#3
				}
			\end{minipage}
		};
	\end{tikzpicture}
	\jonathanskip\vspace{0.05cm}
}

% 2 args: filename, caption
% 1 optional arg: width
\newcommand{\xfig}[3][1]{
    \jonathanskip\vspace{1em}
    \begin{tikzpicture}
        \node[rounded corners=10pt, fill=figcolor, inner sep=10pt] (figbox) {
            \begin{minipage}{0.955\linewidth}
                \centering
                \includegraphics[width=#1\textwidth]{#2}
                \textsf{\figcaption{#3\label{#2}}}
            \end{minipage}
        };
    \end{tikzpicture}
    \jonathanskip\vspace{0.05cm}
}

% 5 args: filename1, caption1, filename2, caption2, overallcaption
% 1 optional arg: width of subfigures
\newcommand{\xfigs}[6][0.45]{
	\jonathanskip\vspace{1em}
	\begin{tikzpicture}
		\node[rounded corners=10pt, fill=figcolor, inner sep=10pt] (figbox) {
			\begin{minipage}{0.955\linewidth}
				\centering
				\begin{tabular}{cc}
					\small\parbox[c]{0.47\linewidth}{\centering\textsf{\textbf{(a)} #3}} & \small\parbox[c]{0.47\linewidth}{\centering\textsf{\textbf{(b)} #5}}\normalsize\\
					\includegraphics[width=#1\linewidth]{#2} & \includegraphics[width=#1\linewidth]{#4}\\
				\end{tabular}
				\textsf{\figcaption{#6\label{#2}}}
				
			\end{minipage}
		};
	\end{tikzpicture}
	\jonathanskip\vspace{0.05cm}
}

% 3 args: filename1, filename2, caption
% 1 optional arg: width of subfigures
\newcommand{\xfigsnocap}[4][0.45]{
	\jonathanskip\vspace{1em}
	\begin{tikzpicture}
		\node[rounded corners=10pt, fill=figcolor, inner sep=10pt] (figbox) {
			\begin{minipage}{0.955\linewidth}
				\centering
				\begin{tabular}{cc}
					\includegraphics[width=#1\linewidth]{#2} & \includegraphics[width=#1\linewidth]{#3}
				\end{tabular}
				\textsf{\figcaption{#4\label{#2}}}
			\end{minipage}
		};
	\end{tikzpicture}
	\jonathanskip\vspace{0.05cm}
}

% 2 args: filename, caption
% 1 optional arg: width of subfigures
\newcommand{\sidefig}[3][1]{
	\jonathanskip\vspace{1em}
	\begin{tikzpicture}
		\node[rounded corners=10pt, fill=figcolor, inner sep=10pt] (figbox) {
			\begin{minipage}[c]{0.5\textwidth}
				\centering
				\includegraphics[width=#1\textwidth]{#2}
			\end{minipage}
			\begin{minipage}[c]{0.455\linewidth}
				\centering
				\textsf{\figcaption{\justifying #3\label{#2}}}
			\end{minipage}
		};
	\end{tikzpicture}
	\jonathanskip\vspace{0.05cm}
}

\newcommand{\sidefigbig}[3][1]{
	\jonathanskip
	\colorbox{figcolor}{
	\begin{minipage}[c]{0.65\textwidth}
  		\centering
  		\includegraphics[width=#1\textwidth]{#2}
	\end{minipage}
	\begin{minipage}[c]{0.325\linewidth}
	 \textsf{\figcaption{#3\label{#2}}}
	\end{minipage}
	}
	\jonathanskip\vspace{0.05cm}
}

\newcommand{\makeshiftfig}[2]{
	\jonathanskip
	\colorbox{figcolor}{
		\begin{minipage}{\linewidth}
			\centering
			
			#1
			
			\textsf{\figcaption{#2}}
			
		\end{minipage}
	}
	\jonathanskip
}

\newcommand{\sidefigr}[3][1]{
	\jonathanskip
	\colorbox{figcolor}{
	\begin{minipage}[c]{0.455\linewidth}
	 \textsf{\figcaption{#3\label{#2}}}
	\end{minipage}
	\begin{minipage}[c]{0.5\textwidth}
  		\centering
  		\includegraphics[width=#1\textwidth]{#2}
	\end{minipage}
	}
	\jonathanskip\vspace{0.05cm}
}

\newcommand{\xfigssmallbig}[6][unused]{
	\jonathanskip
\colorbox{figcolor}{
	\begin{minipage}{0.955\linewidth}
  		\centering
		\begin{tabular}{cc}
		\small\parbox[c]{0.32\linewidth}{\textsf{\textbf{(a)} #3}} & \small\parbox[c]{0.62\linewidth}{\textsf{\textbf{(b)} #5}}\normalsize\\
		\includegraphics[width=0.3\linewidth]{#2} & \includegraphics[width=0.6\linewidth]{#4}\\
		\end{tabular}
		\textsf{\figcaption{#6\label{#2}}}
  	
	\end{minipage}
	
}\jonathanskip\vspace{0.05cm}}

\newcommand{\xfigssize}[8][unused]{
	\jonathanskip
\colorbox{figcolor}{
	\begin{minipage}{0.955\linewidth}
  		\centering
		\begin{tabular}{cc}
		\small\parbox[c]{#7\linewidth}{\textsf{\textbf{(a)} #3}} & \small\parbox[c]{#8\linewidth}{\textsf{\textbf{(b)} #5}}\normalsize\\
		\includegraphics[width=#7\linewidth]{#2} & \includegraphics[width=#8\linewidth]{#4}\\
		\end{tabular}
		\textsf{\figcaption{#6\label{#2}}}
  	
	\end{minipage}
	
}\jonathanskip\vspace{0.05cm}}

\newcommand{\xfigss}[8][0.45\textwidth]{
	%\\[\intextsep]
	\jonathanskip
	\colorbox{figcolor}{
	\begin{minipage}{\linewidth}
  		\centering
		\begin{tabular}{ccc}
		\small\parbox{0.3\linewidth}{\textsf{\textbf{(a)} #3}} & \parbox{0.3\linewidth}{\textsf{\textbf{(b)} #5}} & \parbox{0.3\linewidth}{\textsf{\textbf{(c)} #7}}\normalsize \\
		\includegraphics[width=0.3\linewidth]{#2} & \includegraphics[width=0.3\linewidth]{#4}& \includegraphics[width=0.3\linewidth]{#6}\\
		\end{tabular}
		\textsf{\figcaption{#8\label{#2}}}
  	
	\end{minipage}
	}
	\jonathanskip\vspace{0.05cm}
}

\newcommand{\xfigsss}[9]{
	%\\[\intextsep]
	\jonathanskip
\colorbox{figcolor}{
	\begin{minipage}{0.95\linewidth}
  		\centering
		\begin{tabular}{cccc}
		\small\parbox{0.23\linewidth}{\textsf{\textbf{(a)} #2}} & \small\parbox{0.23\linewidth}{\textsf{\textbf{(b)} #4}} & \small\parbox{0.23\linewidth}{\textsf{\textbf{(c)} #6}} & \small\parbox{0.23\linewidth}{\textsf{\textbf{(d)} #8}} \normalsize\\
		\includegraphics[width=0.23\linewidth]{#1} & \includegraphics[width=0.23\linewidth]{#3} & \includegraphics[width=0.23\linewidth]{#5}& \includegraphics[width=0.23\linewidth]{#7}\\
		\end{tabular}
		
		\textsf{\figcaption{#9\label{#1}}}
	\end{minipage}
	}
	\jonathanskip\vspace{0.05cm}
}

\newcommand{\moviefour}[5]{
	%\\[\intextsep]
	\jonathanskip
\colorbox{figcolor}{
	\begin{minipage}{0.95\linewidth}
  		\centering
			\includegraphics[width=0.21\linewidth]{#1} \hspace{0.1cm} \includegraphics[width=0.21\linewidth]{#2} \hspace{0.1cm} \includegraphics[width=0.21\linewidth]{#3} \hspace{0.1cm} \includegraphics[width=0.21\linewidth]{#4} \hspace{0.1cm}
		%\vspace{-1cm}
		\textsf{\figcaption{#5\label{#1}}}
	\end{minipage}
	}
	\jonathanskip\vspace{0.05cm}
}


\newcommand{\gridfour}[6]{
	%\\[\intextsep]
	\jonathanskip
\colorbox{figcolor}{
	\begin{minipage}{#6\linewidth}
  		\centering
			\includegraphics[width=0.46\linewidth]{#1} \hspace{0.1cm} \includegraphics[width=0.46\linewidth]{#2} \hspace{0.1cm}\\ \includegraphics[width=0.46\linewidth]{#3} \hspace{0.1cm} \includegraphics[width=0.46\linewidth]{#4} \hspace{0.1cm} 
		
		\textsf{\figcaption{#5\label{#1}}}
	\end{minipage}
	}
	\jonathanskip\vspace{0.05cm}
}

\newcommand{\cols}[2]{\parbox{0.5\linewidth}{#1}\parbox{0.5\linewidth}{#2}}
