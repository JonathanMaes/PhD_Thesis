\chapter{Conclusion and outlook}\label{ch:Conclusion}
\glijbaantje{There is no prize to perfection, only an end to pursuit.}{Viktor, \textit{Arcane}}

Throughout this thesis, we have investigated the use of artificial spin ice with out-of-plane magnetisation --- of both the thermally active and non-volatile kind --- as a physical reservoir.
Our interest in OOP ASI for RC was motivated by their well-established input (SOT) and readout (AHE) methods, but their RC potential had yet to be quantified.
To support this work, we developed \hotspice, a flexible CPU/GPU-agnostic Monte Carlo simulator that captures the collective behaviour of large ASI arrays.
Ultimately, three main topics were covered.
\begin{itemize}
	\item The development and underlying model of \hotspice{}, with a focus on several variations of the basic model to improve physical accuracy and algorithmic performance.
	\item The demonstration that thermally active OOP ASI can serve as a reservoir by harnessing its spontaneous relaxation, as evidenced by its ability to perform non-linear signal transformation and prediction.
	\item The design and numerical validation of a two-step ``clocking'' protocol that enables controlled domain wall motion in non-volatile OOP ASI, making them usable for RC with binary input.
\end{itemize}
The key conclusions are summarised below, followed by an outline of possible future research avenues.

\newpage
\paragraph{Summary of main results}
The Ising-like model used by \textbf{\hotspice} to represent single-domain nanomagnets in an ASI allows it to strike a practical balance between physical accuracy and computational efficiency.
Several model variants were proposed that have a limited performance impact but improve the physical accuracy of the simulation.
Firstly, for a more accurate calculation of the magnetostatic interaction, a second-order correction was found to be appropriate for OOP ASI, while a dumbbell model is preferable for IP ASI. % and yields qualitatively different dynamics.
Secondly, different estimates for the effective energy barrier were considered: accounting for asymmetric switching channels proved key to correctly reproduce experimental results for IP ASI, while an exact solution for an idealised energy landscape proved inaccurate for OOP magnets. % Provided that the reduced magnetisation during non-coherent reversal processes was also accounted for.
Finally, two distinct switching algorithms were implemented: the first-switch method is best suited for temporal dynamics, while the Metropolis-Hastings algorithm can efficiently explore the equilibrium state space.
The latter can gain performance by simultaneously sampling multiple sufficiently distant magnets. \par
Using this tool, we showed that \textbf{thermally active} OOP ASI initialised in a uniform state exhibits logarithmic relaxation towards its ground state, as characterised in~\cref{sec:3:relaxation}.
External stimuli can push it away from its ground state, leading to behaviour reminiscent of a leaky integrator.
This allows it to serve as a fading-memory reservoir, as was verified by two benchmark tasks --- non-linear signal transformation and chaotic time-series prediction --- where comparable performance ($\MSE \sim 0.015$) to other physical reservoirs was obtained.
Including a property gradient was found to have a positive impact, as it introduces different relaxation timescales throughout the system, thereby encoding more distinct information.
The logarithmic relaxation leads to a remarkably broad operable frequency window, spanning at least two orders of magnitude. \par
For \textbf{non-volatile} OOP ASI, such a global stimulus would simply saturate the lattice, as avalanches propagate through the system unimpeded.
Therefore, we developed the ``AFM clocking'' protocol: by alternately stimulating two interleaved sublattices, avalanches were avoided and controlled stepwise domain wall motion was achieved in square-lattice OOP ASI.
In the presence of realistic disorder on the energy barrier of magnets, this scheme produces sufficiently complex dynamics for reservoir computing, as the position and size of domains non-linearly encodes recent bits of information.
More broadly, to achieve controlled domain wall movement in an ASI, two key takeaways from the clocking concept are that the input method must lift the degeneracy between the ground state and that at least two independently addressable sublattices must exist to prevent avalanches.  \\\par

Overall, we have demonstrated that out-of-plane artificial spin ice can serve as an effective physical reservoir, with \hotspice serving as a versatile and efficient tool for exploring both ASI physics and optimising ASI for unconventional computing architectures.
\hotspice has proven to be a useful tool for the purposes of this thesis, with potential for use in other settings.
In a broader sense, it can complement micromagnetic simulations to enable the study of complex ASI dynamics in large systems. % TODO: rework this "ending" in the middle

\newpage
\paragraph{Outlook and future directions}
Looking ahead, several avenues can strengthen both the simulation toolkit and the proposed physical reservoir concepts.

\subparagraph{Refinements to \hotspice}
The functionality implemented in \hotspice has grown organically, as it was initially unclear how accurate a basic Ising model would be.
Hence, the various models presented in~\cref{ch:Hotspice} were mostly considered and implemented after the basic code structure had already been established.
This has led to an API that can at times be opaque, with some functionality that has since become obsolete.
Now that the core features are validated, the API can be streamlined to make it more intuitive for prospective users, while retaining the versatility of the current implementation. \par
In this respect, the following enhancements can still be integrated.
Firstly, the usage of Monte Carlo clustering algorithms (e.g., Wolff) can be explored to alleviate critical slowing down, a route which was previously abandoned as it is not straightforward to apply them to the broad range of systems that \hotspice aims to simulate --- i.e., where magnets can have any orientation and with full magnetostatic coupling.
On a similar note, the first-switch method is limited by the fastest-switching magnets in the system, which can lead to a select few magnets repeatedly switching back-and-forth.
A way to alleviate this would be welcome, akin to the MH multi-switching method, but this requires more intricate treatment as time plays a central role in the first-switch algorithm and should not elapse at different rates throughout the ASI. \\\par
%For now, it suffices to use a large number of Monte Carlo sweeps (especially near a critical temperature), but algorithmic improvements remain a welcome addition to \hotspice.

\hotspice represents the ASI on a rectilinear grid, enhancing performance at the cost of restricted positioning of magnets, though many popular lattices can nonetheless be recreated.
For example, the grid enables efficient multi-sampling of magnets in the Metropolis-Hastings algorithm, and allows the efficient calculation of \textit{all} magnetostatic interactions in the system by means of convolutions.
Other codes are often forced to truncate the interaction distance, which we found to incur a non-negligible error on the total magnetostatic energy, exceeding $\SIrange{5}{10}{\percent}$ even when taking the $\approx 1000$ nearest magnets into account.
However, it can be argued that using the Ising model to represent ASI already constitutes a more severe approximation, rendering a truncation justifiable. \\\par

Note that it was ultimately not necessary to port the code to GPU for the purposes of this thesis, as the considered systems were rather small. % except in~\cref{fig:3:Clocking_massive_seeded}
Nonetheless, this effort has expanded the scope of \hotspice for future applications in larger ASI.
One particular bottleneck can still be addressed: currently, determining ensemble averages or sweeping parameters requires a batch of separate simulations.
For small systems, this precludes the use of parallelism on the GPU.
For higher throughput, multiple simulations could be ``stacked'' as 3D arrays, providing opportunities for further parallelisation in large ensembles of small ASIs. \\\par % Does it make sense to determine ensemble averages of small systems? Why not just take a large system? Maybe for edge effects.

While the implemented model variants already make \hotspice usable in a wide range of ASI systems, the aforementioned suggestions could raise the efficiency of \hotspice to the next level.

\subparagraph{Reservoir computing in perpendicular-anisotropy artificial spin ice}
We have successfully used both thermally active and non-volatile out-of-plane ASI as a reservoir, which both required different input protocols.
Still, there are limits to the presented approaches that affect their viability to use them for reservoir computing. \par

While a clocking scheme in non-volatile ASI enables controlled dynamics, it is limited to discrete input and requires intricate manufacturing of conduction pathways.
However, analog data is more prevalent in nature and allows a higher information density, so it is likely desirable to use the full potential of analog input whenever possible.
A couple of approaches to use analog input with a clocking scheme were proposed, but were ultimately dismissed as they did not yield desirable dynamics.
Although an analog clocking scheme is non-trivial to devise, a viable approach may exist and the search for it should not be abandoned.
Thermally active ASI, on the other hand, can readily act as a leaky integrator of analog input. \par
Manufacturing both non-volatile and thermally active ASI reservoirs is not so straightforward.
Non-volatile ASI require intricate layouts of conduction pathways and must avoid conflicts with the readout method, whereas thermally active ASI only require a single conductive underlayer to use them as a reservoir.
Conversely, thermally active nanomagnets are harder to fabricate as great control over the anisotropy is required --- on the order of tens of $\kBT$ --- though downscaling may provide a solution for this as it also lowers the magnetostatic coupling, which is desirable.
% TODO: mention that the achieved energy barriers are far lower than expected, so the reversal process must be better understood
In short, the experimental aspect of these approaches requires further consideration. \\\par

On another note, OOP ASI present less design freedom than IP ASI, as the easy axis of all magnets always has the same orientation.
We therefore only considered the square lattice, as an example of a lattice without nearest-neighbour frustration.
It may be worth considering whether frustrated OOP lattices (e.g., kagome, honeycomb\dots) could exhibit more chaotic dynamics, though the reservoir protocols used here would likely yield very similar results regardless of the precise lattice used, since all OOP lattices have a more or less AFM ground state.
As such, other approaches akin to the property gradient may have more success, where spatial variations are baked into the ASI. \\\par

While OOP ASI are alluring for their ease of electrical interfacing, the aforementioned factors complicate their usage as reservoirs.
Nonetheless, we succeeded in using them as such, with a performance comparable to other types of reservoirs found in literature.
Following this numerical exploration of the RC potential of OOP ASI, an experimental lithographic realisation of the clocking protocol would be desirable, to achieve a better understanding of the performance, feasibility and efficiency in a real environment.


% Furthermore, for the functionality they enable, they take up a non-negligible amount of space on-chip as compared to traditional computer architectures, as the magnets have a size on the order of $\SI{100}{\nano\metre}$ (it is indeed possible to make the magnets smaller, as a high magnetostatic coupling is not required in thermally active ASI, but controlling the OOP anisotropy might be hard), a space in which typically one transistor can be implemented including peripherals like current lines, though one could argue that (thermally active) ASI can react to analog input signals whereas representing floating-point values would require a large number of transistors. % arrays quickly exceed $\SI{1}{\micro\metre}$ in size, a space in which up to 100 transistors could be implemented that perform the same function.

Performance of thermally active ASI was assessed through specific tasks, while non-volatile ASI were assessed through more general metrics (matrix-rank metrics and task-agnostic metrics).
The relation between task-agnostic metrics and the specific tasks that a reservoir will be good at is clearer than for the matrix-rank metrics, making it more interesting to use task-agnostic metrics in future research.
It was, for example, found that decent values of the computing capacity $C$ do not necessarily translate to significantly non-zero memory capacity or non-linearity.



