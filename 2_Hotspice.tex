\chapter{Methods: ``\hotspice'' simulator for ASI}\label{ch:Hotspice}
% \glijbaantje{It's not a bug, it's a feature.}{Someone}

\begin{adjustwidth}{2em}{2em} % TODO END: update once published.
    \begin{center}
        \textbf{Material from this chapter has also been published in:} \\
    \end{center}
    \vspace{1em}
    \begin{adjustwidth}{0em}{3em}
	    \begin{itemize}
	    	\item[\cite{MAES-24}] J.~Maes, D.~De~Gusem, I.~Lateur, J.~Leliaert, A.~Kurenkov, and B.~Van~Waeyenberge.
	    	\newblock The design, verification, and applications of Hotspice: a Monte Carlo simulator for artificial spin ice.
	    	\newblock \emph{ArXiV}, arXiv:\penalty0 2409.05580, 2024.
	    \end{itemize}
    \end{adjustwidth}
    \vspace{0.5em}
    \begin{center}
        \centering\rule{0.7\linewidth}{0.4pt}
    \end{center}
    \vspace{.5em}
    \begin{center}
    	The \hotspice simulator discussed in this chapter\\
    	is open-source and available on \href{https://github.com/bvwaeyen/Hotspice}{GitHub}. \\
    \end{center}
    %\vspace{0.5em}
    \begin{center}
    \centering\rule{0.7\linewidth}{0.4pt}
    \end{center}
    \vspace{1em}
\end{adjustwidth}

To assess the potential of \xref{Reservoir Computing} in (perpendicular-anisotropy) \xref{Artificial Spin Ice}, as is the main topic of this thesis, a simulation framework is needed that allows efficient exploration of the impact that various system parameters and input methods have on the reservoir performance. \\\par

Nanomagnetic systems are often simulated using micromagnetic codes, such as the finite-difference-based \mumax~\cite{mumax3} and \oommf~\cite{OOMMF} or the finite-element-based \nmag~\cite{Nmag}, which capture the magnetization dynamics of individual nanomagnets in great detail. \par
However, the time between successive switches of a nanomagnet is not necessarily similar to the timescale of micromagnetics.
Furthermore, determining RC metrics requires applying many input cycles -- on the order of 100 for the task-agnostic metrics -- to get a statistically valid result.
In this timeframe, many switches occur.
When the simulated time extends beyond several microseconds, as is typically the case, simulating even a modest number of magnets -- on the order of several dozen -- becomes computationally unfeasible~\cite{leo2021chiral}. \\\par

To address these limitations, specialized ASI simulation tools have been developed.
An example of this is the flatspin simulator~\cite{flatspin}, which implements deterministic spin flipping via a Stoner-Wohlfarth model~\cite{StonerWohlfarth2008}.
Using such higher-level approximations enables the study of collective behaviour in much larger systems and over far longer timescales than is feasible with micromagnetic codes, though at the cost of no longer simulating the internal magnetization structure of individual nanomagnets in detail.
Additionally, Monte Carlo methods are often used to simulate spin ices, including ASI, but these are typically specialized to a select few lattice geometries and often only account for nearest-neighbour interactions, whose strength is often arbitrarily set or calculated separately using micromagnetic codes.~\cite{MeltingASI,sklenar2019field,gilbert2014emergent,zhang2013crystallites} \\\par % REFS: 'gilbert2014emergent': MC sims based on a vertex model and interacting magnetic charges. 'zhang2013crystallites': uses monopoles and NN couplings to model kagome ASI with Metropolis and loop update. 'moller2006artificial': bit broader using dipolar interaction, though seemingly still only for nearest neighbors. 'mengotti2011kagome' use a full dipole model with Ewald summation for PBC. 'lou2023competing': dipole model for half-occupation IP Ising. 'sendetskyi2019continuous': dipole model for square ASI. 'EngineeringRelaxationComputation': KMC on dipole model, seemingly for square arrays. 'sklenar2019field': NN interactions calculated by mumax on quadrupole lattice. 'MeltingASI': 16-vertex ice model

Our goal was to blend these two approaches, resulting in \hotspice: a versatile Monte Carlo simulator meant to capture ASI physics with minimal arbitrary parameters, allowing various lattice configurations to be evaluated. \par
This software approximates each single-domain nanomagnet as a single Ising spin, associating energies with the various ASI states by accounting for the magnetostatic interaction between all magnets.
\hotspice supports both in-plane (IP) and out-of-plane (OOP) ASI, which may contain thousands of magnets. Simulations can span arbitrary timescales, as determined by the switching time of magnets in the system. \\\par

In this chapter, we discuss several model variants that have been implemented, and assess their accuracy in simulating the behaviour of ASI.
These variants differ in their calculation of the magnetostatic interactions, the use of symmetric versus asymmetric energy barriers, and their choice of update algorithm. \\\par

\section{Model}
In single-domain IP nanomagnets, the magnetisation prefers to align along the fixed easy axis of the geometry, while for OOP magnets a strong interfacial anisotropy causes a preferential orientation along the $z$-axis.
Either way, for simulations it is natural to use an Ising-like approximation for such single-domain magnets, where the position $\vc{r}_i$, axis $\vc{u}_i$, and size of the magnetic moment\footnote{
	The size of the magnetic moment $\mu_i$ corresponds to the total ground state magnetic moment $\abs{\int_{\Omega_i} \vc{M}(\vc{r})d\vc{r}}$, with $\Omega_i$ the shape of magnet $i$ and $\vc{M}(\vc{r})$ its magnetisation in the twofold degenerate ground state. Due to edge relaxation effects, this value is slightly smaller than $M_\mathrm{sat} V_i$.
} $\mu_i$ of each magnet $i$ are fixed, allowing the magnets to only switch between the `up' ($\uparrow$) and `down' ($\downarrow$) states.
Thus, the total magnetic moment vector of magnet $i$ can be expressed as
\begin{equation}
	\vc{\mu}_i = s_i \mu_i \vc{u}_i \mathrm{,}
\end{equation}
where $s_i = \pm 1$ and $\abs{\vc{u}_i} = 1$. \par
The switching rate between these two states is determined by the temperature $T$ and the effective energy barrier $\widetilde{E_\mathrm{B}}$ separating them.
%A canonical ensemble is used where magnets are considered in contact with a heat bath of temperature $T_i$. % TODO: mention this or not?
For an isolated nanomagnet, the energy barrier $E_\mathrm{B} = K_\mathrm{u} V$\footnote{
	This is valid for switching by coherent rotation. Similar to the calculation of $\mu$, edge relaxation effects may cause the effective volume to be slightly smaller.
} originates from its uniaxial shape anisotropy $K_\mathrm{u}$.
Interactions with other magnets or external fields modify the energy landscape, leading to an effective barrier $\widetilde{E_\mathrm{B}}$, which we will denote with a tilde.~\cite{leo2021chiral}
Each magnet can have a unique magnetic moment size $\mu_i$, temperature $T_i$ and energy barrier $E_{\mathrm{B},i}$.
This enables, for instance, modelling some of the disorder due to lithographic variations by assigning a different shape anisotropy to each magnet, typically sampled from a Gaussian distribution with mean $E_\mathrm{B}$ and standard deviation $\sigma(E_\mathrm{B})$. \\\par

Due to the periodic nature of many ASI lattices, \hotspice chooses to perform the simulation on a rectilinear grid, the benefits and details of which will be discussed in~\cref{sec:2:Implementation}.
Each grid point may or may not contain a magnet, and the magnets must either all be of the IP type, or all OOP.
Even though this implementation does not allow complete freedom in the placement of magnets, many popular ASI lattices can be constructed in this manner. \par

\xfig[1.0]{2_Hotspice/ASIs.pdf}{
	\label{fig:2:ASIs}Predefined artificial spin ice (ASI) lattices available in \hotspice. The unit cell of each lattice is delineated by a central dark grey rectangle. The red indicator defines the lattice parameter $a$. In the Ising approximation, the magnetization of in-plane magnets (top) aligns along the major axis of the depicted ellipses. Out-of-plane magnets (bottom) are illustrated as circles.
}

\cref{fig:2:ASIs} showcases the 12 lattices that \hotspice provides out-of-the-box. The pinwheel lattices (a) and (b) are equal to the square lattices (c) and (d), respectively, but with each magnet individually rotated by \ang{45}. Both lattices have two variants (diamond/lucky-knot and closed/open), related by a global \ang{45} rotation of the entire lattice. This gives rise to different boundaries due to the Cartesian character of the underlying grid, which may alter the dynamics of the ASI. Furthermore, the unit cell for lucky-knot pinwheel and open square is more compact, resulting in faster simulation, while diamond pinwheel and closed square are more popular in literature and were therefore implemented separately. \par % TODO: more popular in literature? Examples where these are used?
The remaining four IP and four OOP lattices are also related: the magnets in the OOP lattices (i)-(l) are positioned at the vertices where magnets meet in their respective IP counterparts (e)-(h). \\\par


\section{Energy calculation}
\subsection{Energy terms}
Three energy contributions have been implemented in \hotspice, supporting both open and periodic boundary conditions (PBC).\footnote{
	Users can implement more energy contributions by inheriting from the \python{hotspice.Energy} class and implementing the \python{abstractmethod}s, taking care to correctly account for open or periodic BC when necessary.
}
\begin{enumerate}
	\item The \textit{magnetostatic interaction energy} between magnets $i$ and $j$
	\begin{equation}
		E_{\mathrm{MS},i,j} = \frac{\mu_0}{4 \pi} \ab(\frac{\vc{\mu}_i \bcdot \vc{\mu}_j}{\abs{\vc{r}_{ij}}^3} - \frac{3(\vc{\mu}_i \bcdot \vc{r}_{ij}) (\vc{\mu}_j \bcdot \vc{r}_{ij})}{\abs{\vc{r}_{ij}}^5}) \mathrm{,}
		\label{eq:E_MS}
	\end{equation}
	with $\mu_0$ the vacuum permeability and $\vc{r}_{ij} = \vc{r}_i - \vc{r}_j$ the vector connecting the two magnetic dipoles $\vc{\mu}_i$ and $\vc{\mu}_j$. \par
	This is the main interaction dictating how nanomagnets influence each other, causing the typical properties of the various ASI lattices, e.g. superferromagnetism in the pinwheel lattice~\cite{li2018pinwheel}.
	Because of its importance, this is the only interaction \hotspice considers by default when an ASI is created.
	Any other energy contributions must explicitly be added to an ASI; see \cref{sec:2:API_energies}.
	This avoids wasting calculations on energies not relevant to the simulation.
	
	\item The \textit{Zeeman energy} of an external field $\vc{B}_\mathrm{ext}$ interacting with magnet $i$
	\begin{equation}
		E_{\mathrm{Z},i} = -\vc{\mu}_i \bcdot \vc{B}_\mathrm{ext} \mathrm{,} \label{eq:E_Z}
	\end{equation}
	where $\vc{B}_\mathrm{ext}$ can be set for each magnet individually. \par
	This energy contribution provides a means for the outside world to interact with the system, and is therefore indispensable when we will be investigating reservoir computing later on.
	Even if input is provided through other means than an external field, this energy contribution can often still be used by considering an effective field instead.
	
	\item The \textit{exchange coupling energy} between nearest neighbours (NN) $i$ and $j$
	\begin{equation}
		E_{\mathrm{exch},i,j} = J \frac{\vc{\mu}_i \bcdot \vc{\mu}_j}{\mu_i \mu_j} \mathrm{,} \label{eq:Eexch}
	\end{equation}
	with $J$ the exchange coupling constant, which is constant throughout the ASI. \par
	This interaction is rarely present in ASI, but can for example be relevant in interconnected ASI -- whether by design or due to limited lithographic accuracy.
	We will encounter an example of the latter in \cref{sec:3:OOP:MFM}.
\end{enumerate}

The combined \textit{interaction energy} $E_i$ of a single magnet $i$ with its environment is then given by
\begin{equation}
	E_i = E_{\mathrm{Z},i} + \sum_j E_{\mathrm{MS},i,j} + \sum_{j \in \mathcal{N}_i} E_{\mathrm{exch},i,j} \mathrm{,}
	\label{eq:E}
\end{equation}
where $\mathcal{N}_i$ is the collection of nearest neighbours of magnet $i$.
Which magnets are included in this collection depends on the ASI lattice and which site of the unit cell magnet $i$ is in, and can be defined separately for each ASI lattice. \\\par
Note that all terms in \cref{eq:E} simply change sign when magnet $i$ switches ($\vc{\mu}_i \rightarrow -\vc{\mu}_i$).
The magnetostatic self-energy is not present in \cref{eq:E} because it only contributes a constant offset and does not change when the magnet switches.
As such, $E_i$ represents the total interaction energy of a magnet with its `neighbours',\footnote{
	In this context, a `neighbour' of a magnet can be interpreted more broadly as all magnets it interacts with through a particular energy contribution. For example, the magnetostatic interaction considers all magnets to be `neigbours', unless the user has explicitly set a maximum interaction distance.
} and therefore the change in energy of the whole ASI when magnet $i$ switches is simply $\Delta E_{i,1\rightarrow2} = -2 E_i$.
This is called the \textit{switching energy}, and we will later see (\cref{sec:2:Dynamics}) that it takes a central role in the algorithms used for simulating system dynamics.
It is therefore very advantageous to have such a simple method of calculating the switching energy. \par
When the simulation is initialized, the energy contributions are calculated for all magnets.
Each magnet therefore has a value for each of the terms in \cref{eq:E} stored in memory.
Whenever a magnet switches, the energies of its `neighbours' are updated appropriately, which constitutes a major part of the calculation effort required for every step in the simulation.

\subsection{Finite-size correction for magnetostatic energy}
\subsubsection{Second-order correction for dipoles}
\subsubsection{Dumbbell model}
\subsubsection{Comparison}
\subsection{Effective energy barrier}
\subsubsection{Intrinsic barrier from shape anisotropy}
\subsubsection{Mean-barrier approximation} % and relevant choices with discontinuities
\subsubsection{Asymmetric barrier}
\subsubsection{Exact solution} % OOP only

\section{Dynamics}\label{sec:2:Dynamics}
\subsection{N\'eel relaxation: temporal evolution}
\subsection{Metropolis-Hastings: sampling equilibrium states}
% TODO: decide whether to talk about nomenclature details etc. here, or in the introduction.
\subsection{Other Monte Carlo algorithms} % Not implemented, explain why Wolff could be useful but why it's not in Hotspice

\section{Implementation}\label{sec:2:Implementation}
\subsection{Package structure} % Structure of Hotspice, explain all modules (core, ASI, energies, (config? with GPU), io, experiments, gui and mostly deprecated plottools, utils but only those relevant to users)
Hotspice is written as a Python 3.10 package and can perform simulations on either the CPU or GPU.
The optimal hardware choice depends on the size of the ASI and the update scheme used.
By default, \hotspice runs on the CPU using the popular NumPy and SciPy libraries.
For GPU-accelerated array manipulation, the CuPy v11.4~\cite{CuPy} library is used. \par
\paragraph{\python{hotspice.config}: GPU/CPU choice}
By default, \hotspice runs on the CPU. To run on the GPU, the environment variable \python{HOTSPICE_USE_GPU} must be set to \python{"true"} before the the \hotspice package is loaded in a Python script with \python{import hotspice}.
\subsubsection{ASI}
\paragraph{\python{hotspice.ASI}: predefined ASI lattices}
This module provides two abstract classes from which all ASIs should inherit: \python{hotspice.ASI.IP_ASI} or \python{hotspice.ASI.OOP_ASI}.
Various lattices are available, as previously shown in~\cref{fig:2:ASIs}.
These all follow the same pattern: \python{hotspice.ASI.<ASI_name>(a, n, **kwargs)}, with two positional arguments.
The first argument is the lattice parameter, as defined by the red indicator in \cref{fig:2:ASIs}.
The second argument is the size of the underlying grid (one can also specify \python{nx} and \python{ny} separately).
For the predefined lattices in the \python{hotspice.ASI} module, these two parameters are enough information to create a rudimentary ASI.
\paragraph{Energies}\label{sec:2:API_energies}
Three predefined energy contributions are provided in the \python{hotspice.energies} module, though they can be accessed from the main \python{hotspice} namespace because they are used so often.
The magnetostatic interaction is implemented by the \python{hotspice.DipolarEnergy} or \python{hotspice.DiMonopolarEnergy} classes, with the latter using the monopole approximation to calculate the interaction.
By default, an ASI object takes only the \python{hotspice.DipolarEnergy} into account.
When relevant, the user has to explicitly add a \python{ZeemanEnergy} or \python{ExchangeEnergy}.
It is possible to set longer-range exchange interactions (next-nearest neighbour etc.) by manually setting the \python{local_interaction} field of an \python{ExchangeEnergy} object.
% TODO: create a custom "API" environment that we can put near all the different parts of the model to show code outside the main text?
\subsubsection{RC}
\subsection{Performance} % Several factors impacting performance, explain by using that one graph
The performance of \hotspice has been improved throughout development in various ways. As previously mentioned, the size of the ASI is the main factor determining whether calculation on the GPU or CPU will result in a faster simulation.

We will start with a summarizing table of these various improvements, and explain them in more detail in the following sections.
Two tables are shown: one for the initialisation time, and one for the time it takes to perform 5000 switches.
The initialisation encompasses the calculation of the initial energy $E_i$ of all magnets, and lookup tables we call `kernels' (see~\cref{sec:2:Kernels}) which serve to reduce the amount of calculations required for every switch.

This test was benchmarked on an NVIDIA GeForce RTX 3080 Laptop GPU; displayed simulation times serve an illustrative purpose. For other GPU models, absolute values may vary significantly, while relative values within the tables should remain mostly similar.

\xtable[tab:2:perf_init]{\textbf{Initialization time} for various simulation sizes $L$ (i.e. $L \times L$ cells). `Mem' indicates that the size $L$ requires too much memory, while `?' means it takes over \SI{1000}{\second}.}{
	\begin{tabular}{r|c|c|c|c|c|c}
		Simulation size $L$ & 50 & 100 & 150 & 200 & 400 & 1000 \\
		\hline \hline
		\makecell{Pure NumPy, full kernel\\ \python{Dipolar_interaction}} & 0.5s & 6.8s & 34.0s & Mem & Mem & Mem \\
		\hline
		\makecell{CuPy, full kernel\\ \python{Dipolar_interaction}} & 2.2s & 5.4s & 11.9s & Mem & Mem & Mem \\
		\hline
		\makecell{CuPy, re-calculating\\single-switch update} & 1.0s & 3.4s & 7.6s & 14.5s & 91.6s & ? \\
		\hline
		\makecell{CuPy, precalculating\\for a unit cell (1D index)} & 2.4s & 4.6s & 9.7s & 16.3s & 66.2s & 730.4s \\
		\hline \hline
		\makecell{CuPy, 2D indexing and\\convolution initialization} & 1.0s & 1.0s & 1.0s & 1.0s & 1.6s & 23.9s \\
		\hline
		\makecell{NumPy, 2D indexing and\\convolution initialization} & 0.014s & 0.2s & 1.0s & 3.2s & 53.1s & ? \\
		\hline
	\end{tabular}
}

\xtable[tab:2:perf_switch]{\textbf{5000 switches time} for various simulation sizes $L$ (i.e. $L \times L$ cells). `Mem' indicates that the size $L$ requires too much memory, while `?' means it takes over \SI{1000}{\second}.}{
	\begin{tabular}{r|c|c|c|c|c|c}
		Simulation size $L$ & 50 & 100 & 150 & 200 & 400 & 1000 \\
		\hline \hline
		\makecell{Pure NumPy, old kernel\\ \python{Dipolar_interaction}} & 9.4s & 217.6s & \makecell{I don't even\\ wanna know} & Mem & Mem & Mem \\
		\hline
		\makecell{CuPy, old kernel\\ \python{Dipolar_interaction}} & 5.0s & 13.6s & 51.8s & Mem & Mem & Mem \\
		\hline
		\makecell{CuPy, re-calculating\\single-switch update} & 7.1s & 7.2s & 8.2s & 9.0s & 13.0s & ? \\
		\hline
		\makecell{CuPy, precalculating\\for a unit cell (1D index)} & 7.8s & 7.9s & 8.3s & 8.9s & 11.8s & 36.9s \\
		\hline \hline
		\makecell{CuPy, 2D indexing and\\convolution initialization} & 6.8s & 7.0s & 7.2s & 7.6s & 10.2s & 35.9s \\
		\hline
		\makecell{NumPy, 2D indexing and\\convolution initialization} & 0.56s & 1.3s & 2.475s & 4.238s & 41s & ? \\
		\hline
	\end{tabular}
}

\subsubsection{GPU vs. CPU}
\subsubsection{Kernels}\label{sec:2:Kernels} % If not yet explained in detail earlier, talk about the perpendicular kernel etc. here
\paragraph{Numerical error with cut-off kernel}
\subsubsection{Multi-switching in Metropolis-Hastings}
\paragraph{Minimal distance between sampled magnets} % Derive equation
\paragraph{Poisson disc}
\paragraph{Grid-select}
\paragraph{Hybrid grid-poisson}
\subsubsection{RNG} % Little to no impact iirc, can be omitted probably
\subsection{Advantages and disadvantages of this approach} % Hindsight is 20/20
\subsubsection{Grid}
\subsubsection{Input/output RC}

\section{Verification} % See paper
\subsection{Hexagon}
\subsection{Non-interacting spin ensemble}
\subsection{Exchange-coupled OOP square system}
\subsubsection{Critical slowing down}
\subsection{Exchange- and magnetostatically-coupled OOP square system}
\subsection{Square-to-pinwheel transition angle}
