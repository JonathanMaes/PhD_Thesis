\chapter{Methods: ``\hotspice'' simulator for ASI}\label{ch:Hotspice}
% \glijbaantje{It's not a bug, it's a feature.}{Someone}

\begin{adjustwidth}{2em}{2em} % TODO END: update once published.
    \begin{center}
        \textbf{Material from this chapter has also been published in~\cite{MAES-24}:} \\
    \end{center}
    \vspace{1em}
    J.~Maes, D.~De~Gusem, I.~Lateur, J.~Leliaert, A.~Kurenkov, and B.~Van~Waeyenberge.
    \newblock The design, verification, and applications of Hotspice: a Monte Carlo simulator for artificial spin ice.
    \newblock \emph{ArXiV}, arXiv:\penalty0 2409.05580, 2024.
    \vspace{1em}
    \begin{center}
        \centering\rule{0.6\linewidth}{0.4pt}
    \end{center}
    \vspace{1em}
\end{adjustwidth}

To assess the potential of \xref{Reservoir Computing} in (perpendicular-anisotropy) \xref{Artificial Spin Ice}, i.e. the central goal of this thesis, simulations will be required to efficiently explore the impact of system parameters on performance. Nanomagnetic systems are often simulated using micromagnetic codes, such as the finite-difference-based \mumax~\cite{mumax3} and \oommf~\cite{OOMMF} or the finite-element-based \nmag~\cite{Nmag}, which capture the magnetization dynamics of individual nanomagnets in great detail. However, the time between successive switches of a nanomagnet is not necessarily similar to the timescale of micromagnetics: when the simulated time extends beyond several microseconds, simulating even a modest number of magnets -- on the order of several dozen -- becomes computationally unfeasible~\cite{leo2021chiral}. \par % Q: is it ok to just copy sentences from my paper?

% TODO: this is placeholder text to test some commands.
Hotspice is written in Python 3.10, and can choose to calculate on either the CPU or GPU.
CuPy v11.4~\cite{CuPy} is used for GPU-accelerated array manipulation, otherwise NumPy and SciPy are used for the CPU.
When writing code for the GPU, it is important to keep the \href{https://docs.nvidia.com/cuda/cuda-c-best-practices-guide/index.html}{Cuda best practices} in mind.
Unfortunately, I only found this page after one year of Hotspice development *shrug*.
Another Python package for parallel programming is the \python{cuNumeric} package by Nvidia, which is also designed to be a drop-in \python{numpy} replacement but supports multiple GPUs, which can be useful for very large simulations that do not fit into the memory of a single GPU.
There is also the \code{nvc++} compiler but this is for Cython code. \par

\section{Model}

\cite{Hotspice}

\xfig[1.0]{2_Hotspice/ASIs.pdf}{\label{fig:2:ASIs}Predefined artificial spin ice (ASI) lattices available in \hotspice{}. The unit cell of each lattice is delineated by a central grey rectangle. The red indicator defines the lattice parameter $a$. In the Ising approximation, the magnetization of in-plane magnets (left) aligns along the major axis of the depicted ellipses. Out-of-plane magnets (right) are illustrated as circles.}\\

\section{Energy calculation}
\subsection{Energy terms}
\subsection{Finite-size correction for magnetostatic energy}
\subsubsection{Second-order correction for dipoles}
\subsubsection{Dumbbell model}
\subsubsection{Comparison}
\subsection{Effective energy barrier}
\subsubsection{Intrinsic barrier from shape anisotropy}
\subsubsection{Mean-barrier approximation} % and relevant choices with discontinuities
\subsubsection{Asymmetric barrier}
\subsubsection{Exact solution} % OOP only

\section{Dynamics}
\subsection{N\'eel relaxation: temporal evolution}
\subsection{Metropolis-Hastings: sampling equilibrium states}
% TODO: decide whether to talk about nomenclature details etc. here, or in the introduction.
\subsection{Other Monte Carlo algorithms} % Not implemented, explain why Wolff could be useful but why it's not in Hotspice

\section{Implementation details}
\subsection{API} % Structure of Hotspice, explain all modules
\subsubsection{ASI}
\subsubsection{RC}
\subsection{Performance} % Several factors impacting performance, explain by using that one graph
\subsubsection{GPU vs. CPU}
\subsubsection{Kernels} % If not yet explained in detail earlier, talk about the perpendicular kernel etc. here
\paragraph{Numerical error with cut-off kernel}
\subsubsection{Multi-switching in Metropolis-Hastings}
\paragraph{Minimal distance between sampled magnets} % Derive equation
\paragraph{Poisson disc}
\paragraph{Grid-select}
\paragraph{Hybrid grid-poisson}
\subsubsection{RNG} % Little to no impact iirc, can be omitted probably
\subsection{Advantages and disadvantages of this approach} % Hindsight is 20/20
\subsubsection{Grid}
\subsubsection{Input/output RC}

\section{Verification} % See paper
\subsection{Hexagon}
\subsection{Non-interacting spin ensemble}
\subsection{Exchange-coupled OOP square system}
\subsubsection{Critical slowing down}
\subsection{Exchange- and magnetostatically-coupled OOP square system}
\subsection{Square-to-pinwheel transition angle}
